\documentclass[10pt,a4paper,titlepage,twoside]{report}

\usepackage{style}

\begin{document}

\title{Utvärdering av JSON Scheman som verktyg för att generera grafiska användargränssnitt för datamanipulation}
\alttitle{Evaluation of JSON Schemas as a method of generating graphical user interfaces for data manipulation}
\author{Julius Recep Colliander Celik}
\email{jcelik@kth.se}
\supervisor{Patric Dahlqvist}
\examiner{Anders Västberg}
\principal{LS Elektronik AB}
\programme{Civilingenjör Informationsteknik}
\school{Skolan för elektroteknik och datavetenskap}
\date{\today}

% Frontmatter includes the titlepage, abstracts and table-of-contents
\frontmatter

%\titlepage

\maketitle
\cleardoublepage
\newpage\leavevmode\thispagestyle{empty}
\newpage\leavevmode\thispagestyle{empty}

\setlength{\parindent}{0pt}

\setlength{\parskip}{\baselineskip}

\section*{Sammanfattning}
Det saknas en bra lösning för att automatiskt generera inmatningsformulär för datamanipulation. Därför måste grafiska användargränssnitt skapas utifrån förutbestämda kända faktorer om datan som det grafiska användargränssnittet ska baseras på. Det här arbetet använder JSON Schema för att beskriva data som kan manipuleras och använder det för att automatiskt generera grafiska användargränssnitt. Genom att använda samma modell kan grafiska användargränssnitt genereras generellt utan att i förväg ha information om datan som det grafiska användargränssnittet ska baseras på.

Arbetet är en fallstudie som ämnar att lösa ett specifikt problem, samtidigt som JSON Schema som verktyg utvärderades ur ett generellt perspektiv. Dessutom utvärderar arbetet en stor mängd befintliga verktyg som använder JSON Scheman för att generera inmatningsformulär. En generell lösning för att generera inmatningsformulär utifrån JSON Scheman presenteras i arbetet och förbättringsförslag för JSON Scheman föreslås. 

Slutsatsen som drogs om JSON Scheman var att det är ett bra verktyg för ändamålet, men har brister, vilket kan förväntas då JSON Schema inte var en färdig teknik, vid tidpunkten då arbetet genomfördes. Bristerna gällde annotering, medan att använda JSON Scheman för valideringssyften ansågs fungera utmärkt. De mest kritiska bristerna skulle kunna åtgärdas om nyckelordet \textit{format} utökades med att bidra mer generell annoteringsinformation än en textsträng, helst med nyckelordet \textit{format} som objekt, samt att enumvektorelement erbjöd utökade annoteringsmöjligheter, i form av metainformation på elementnivå.

\clearpage

\section*{Abstract}
There does not exist a good solution to automatically generate input forms for data manipulation. Therefore user interfaces has to be created considering predetermined known factors about the data that the user interface will be based upon. This project uses JSON Schema to describe data that can be manipulated and uses that to automatically generate user interfaces. By using the same model unspecific graphical user interfaces can be generated without prior information about the data that the user interface will be based upon.

The study is a case study that aims to solve a specific problem, while simultaneously evaluating JSON Schema as a tool from a general context. Furthermore the study evaluates a large set of existing tools that use JSON Schemas to generate input forms. A general solution to generating input forms based on JSON Schemas is presented in the study and potential improvements on JSON Schema are suggested.

The conclusion that was drawn about JSON Schema was that it is a good tool for the purpose, however there are inadequacies, which could be expected as JSON Schema was not a finalized technology, at the time this study was conducted. The inadquacis regarded annotation, while using JSON Schema for validation purposes was regarded as working excellently. The most critical faults could be corrected if the keyword \textit{format} was extended by offering more general annotation information than a string, preferably with the keyword \textit{format} as an object, as well as enum vector elements offering more extensive annotation options, in the form of metainformation on element level.

\clearpage

\setlength{\parskip}{0pt}

\tableofcontents

\setlength{\parskip}{\baselineskip}


% Mainmatter is where the actual contents of the thesis goes
\mainmatter

\chapter{Introduktion}

Examensarbetet handlar om att annotera manipulerbar data, och sedan automatiskt generera ett grafiskt användargränssnitt för att interagera med datan. Vad jag känner till är detta det första publikt publicerade försöket av att generera ett grafiskt användargränssnitt från JSON Schema, som var implementerat i något annat programmeringsspråk än HTML, CSS eller JavaScript. Detta är relevant då JSON Scheman utvecklats med hänsyn till JSON och JavaScript, samtidigt som många andra språk och miljöer skulle kunna dra nytta av funktionaliteten som JSON Schema kan erbjuda. Det är viktigt att JSON Schema fungerar bra med andra språk än JavaScript för att det ska vara ett bra och användbart verktyg.

\section{Bakgrund}
\label{sec:intro:bakgrund}
Det här kapitlet beskriver bakgrunden till varför och i vilka ämnesområden arbetet utfördes, samt för att ge en förståelse för resten av introduktionen. JSON och JSON Schema utgör en stor del av teknikerna som arbetet undersökte. LS Elektronik AB \textit{(LSE)} är företaget arbetet utfördes hos, och Mimer SoftRadio är systemet som arbetet utvecklades mot.

\subsection{JSON och JSON Schema}
\label{sec:intro:json}
JavaScript Object Notation \textit{(JSON)} är ett textbaserat dataformat för att utbyta data mellan webbtjänster. Till skillnad mot andra alternativ, som exempelvis XML, är det både läsbart för människor och datorer, samtidigt som det är väldigt kompakt, vilket är en anledning till att det är ett av de mest populäraste dataformaten för datautbyte mellan webbtjänster \cite{Pezoa2016}. JSON utvecklades med inspiration till språket ECMAScript \textit{(JavaScript)} men är samtidigt oberoende av programmeringsspråk, vilket lett till att implementationer för att generera och parsa JSON finns tillgängliga i många olika programmeringsspråk \cite{ECMA2013}. ECMAScript är ett språk som stöds av alla moderna webbläsare, och har därför blivit en kärnteknik för webben.

Trots att JSON är det populäraste dataformatet för datautbyte mellan webbtjänster saknas det ett väletablerat standardiserat ramverk för metadata-definition \cite{Pezoa2016}. En väldigt lovande formell standard är JSON Schema, vilket är ett ramverk som fortfarande utvecklas av Internet Engineering Task Force \textit{(IETF)}. JSON Schema är ett ramverk för att beskriva och annotera JSON-data \cite{A.Wright}. Kapitel \ref{sec:teori} beskriver JSON och JSON Schema i mer detalj.

\subsection{Mimer SoftRadio}
\label{sec:intro:mimer}
Arbetet utfördes hos LSE, som är ett teknikföretag, som utvecklar och tillverkar elektroniska produkter \cite{Ehne}. LSE erbjuder bland annat ett radiosystem som heter Mimer SoftRadio vilket kan användas för att ansluta ett flertal annars inkompatibla radioenheter i ett och samma system, samt fjärrstyra radioenheterna från en persondator med ett klientprogram. I resten av rapporten kan datorn med klientprogrammet kallas operatörsdator, där användaren kan kallas operatör.

Mimer SoftRadio \textit{Mimer} är ett program med väldigt många möjliga inställningar. I många fall är dessa inställningar för komplexa för de vanliga operatörerna att redigera själva, så därför brukar vissa kunder låsa redigeringsmöjligheterna, och bara tillåta vissa administratörer att ställa in alla inställningar på rätt sätt. Det finns också kundfall där flera operatörer använder samma dator, vid olika tidpunkter. Ett förekommande kundfall är att en operatör jobbar dagtid med att leda och organisera dagsarbete, medan en annan operatör tar över nattskiftet för att övervaka många fler radioenheter.

För att förenkla dessa två kundfall påbörjade LSE utvecklingen av funktionalitet som skulle erbjuda användare att spara uppsättningar av inställningar i olika \textit{profiler}. Det skulle gå att enkelt byta mellan flera förinställda konfigurationer av Mimer. För att konfigurera dessa profiler skapades ett administratörsprogram, som skulle kunna fjärrkonfigurera profilerna hos operatörsdatorerna. Fjärrstyrningen skulle underlätta administratörer att ställa in profiler på flera datorer samtidigt, som sannolikt skulle innehålla liknande inställningar.

\section{Problemområde}
\label{sec:intro:problemområde}
Systemet för att konfigurera profiler (se kapitel \ref{sec:intro:mimer}) bygger på att alla operatörsdatorer exponerar ett API mot en server, över en TCP-port. Administratörsprogrammet ska erbjuda ett grafiskt användargränssnitt för att konfigurera profilinställningar, för att sedan kommunicera ändringarna till operatörsdatorerna. I resten av rapporten kan operatörsdatorerna och administratörsprogrammet kallas server respektive klient. Administrationsprogrammet skrivs som en skrivbordsapplikation till Windows, med språket Delphi.

Kommunikationsprotokollet är ett eget skapat protokoll som bygger på att skicka JSON-objekt via TCP. För en beskrivning av JSON, se kapitel \ref{sec:teori:json}. För en mer utförlig beskrivning av kommunikationsprotokollet se bilaga \ref{appendix:api-beskrivning}. Problemet som LSE har inför utvecklandet av användarprofilerna är skapandet av ett grafiskt användargränssnitt på administratörsprogrammet. Olika operatörsdatorer, hos olika kunder, kan ha olika versioner av Mimer, med olika funktionalitet tillgänglig, och därmed olika uppsättningar konfigurerbara inställningar.

Det vore orimligt kostsamt för LSE att skapa olika administratörsprogram för varje version av Mimer, då både Mimer ändras med tiden, samt att olika kunder köper till extra funktionalitet. Samtidigt måste det grafiska användargränssnittet på administratörsprogrammet anpassas så att det ska vara tydligt vad en administratör kan konfigurera. Helst ska ett administratörsprogram fungera bra med framtida versioner av Mimer, utan några eller utan stora justeringar av programmet. LSE vill helt enkelt att servern kommunicerar tillgängliga inställningar till klienten, så att klienten sen ska kunna anpassa sitt grafiska användargränssnitt, och det ska ske på ett så tillräckligt generellt sätt att administratörsprogrammet är framtidssäkert för framtida versioner av Mimer.

\section{Problem}
\label{sec:intro:problem}
Vilka svårigheter finns det med att använda JSON Schema för att automatisk generera ett grafiskt användargränssnitt? Är det möjligt? Hur generella JSON Scheman går det att använda sig av?

Det finns även mer praktiska problem som uppstår vid konstruktionen av arkitekturen för systemet som ska utvecklas. De är:

\begin{itemize}
	\item Hur ska användarinteraktion integreras i detta system, så att interaktion faktiskt manipulerar data?
	\item Hur ska schemat som formuläret baseras på, genereras?
	\item Hur ska schemat parsas, tolkas och sedan representeras i språket Delphi?
	\item Hur ska schemats representation tillsammans med data presenteras i ett grafiskt användargränssnitt?
\end{itemize}

\section{Syfte}
\label{sec:intro:syfte}
Syftet med rapporten är att systematiskt analysera problemen med att försöka skapa automatiskt genererade grafiska användargränssnitt utifrån olika JSON Scheman. Syftet med arbetet är att med hjälp av JSON Scheman skapa ett dynamiskt grafiskt användargränssnitt som anpassade sig efter syfte. Utan att uppdatera administratörsklienten ska samma administratörsklient kunna konfigurera inställningar hos olika datorer med olika versioner av Mimer, och därmed olika uppsättningar konfigurerbara inställningar. Det skulle inte bara innebära stark kompatibilitet utan också framtidssäkerhet hos LSEs produkter.

\section{Mål}
\label{sec:intro:mål}
Målet med arbetet är att kunna skapa en grund för användare av Mimer att enkelt kunna konfigurera inställningar, oavsett version eller uppsättning extra funktionalitet. Det skulle kunna innebära att Mimer blir ett bättre verktyg för många potentiella kunder. Samtidigt finns det ett mål med att utforska samt att metodiskt utvärdera och beskriva hur grafiska användargränssnitt för att redigera data, automatiskt kan genereras.

\subsection{Samhällsnytta, Etik och Hållbarhet}

Ur ett samhällsnyttigt och etiskt perspektiv kan en implementation av fjärrstyrda användarprofiler i Mimer SoftRadio innebära effektivisering av samhällsnyttiga funktioner. Mimer användas bland annat av polis, ambulans, brandkår, kollektivtrafik och internationella flygplatser. Fjärrstyrda användarprofiler skulle hos befintliga kunder i många fall innebära effektivare arbete. Som följd av detta går det att argumentera för att det leder till effektivare kommunikation för organisationer som använder Mimer. Då dessa organisationer arbetar med säkerhet i samhället, räddandet av liv, och upprätthållandet av ett effektivt samhälle, lider hela samhället när dessa organisationer inte kan kommunicera ordentligt. Det är därför väldigt etiskt försvarbart att arbeta med att effektivisera arbetet hos dessa organisationer.

\section{Risker}
\label{sec:intro:risker}
En ekonomisk risk är risken som alltid finns vid all hantering av data. Om någon datahantering skulle bli fel, och data skulle försvinna, skrivas över eller bli korrupt, så måste kunder tillägna tid åt att återskapa datan. Därför är det viktigt att implementera ett robust system som är delvis feltolerant. Ingen data som kommer hanteras kommer vara kritisk, och kommer vara relativt enkel att återskapa. Problemet blir att det skulle innebära en kostnad att behöva skapa profiler igen och ställa in inställningar igen, och utöver det så skulle korrupta filer till och med kunna innebära att Mimer SoftRadio inte går att använda alls.

\section{Metodval}
\label{sec:intro:metodval}
Arbetet var av kvalitativ karaktär. Arbetet handlade om att utforska JSON Scheman som är en ny utvecklande teknik. Specifikt utforskades förmågan att använda JSON Scheman för annoteringar som sedan används för att generera grafiska användarformulär för att låta användare manipulera den data som annoterades. 

Det saknas stora datamängder som krävs för att genomföra statistisk analys, som är kärnan av kvantitativa studier. Dessutom är JSON Schema som ämne fortfarande outforskat, och speciellt att använda JSON Scheman för att generera grafiska användargränssnitt för att manipulera datan som schemat beskriver.

Problemet som arbetet grundade sig på var en fallstudie. Ett specifikt problem skulle lösas utifrån vissa krav och en ny, oprövad och ofärdig teknik skulle användas för att lösa problemet. Det fanns vissa befintliga lösningar som utvärderades utifrån perspektivet av fallstudien. Både de befintliga lösningarna utvärderades, samt hur användbart JSON Schema som verktyg är för att lösa problemet som presenterades i fallstudien.

För att utvärdera om någon av lösningarna löste fallstudien, utvärderades alla befintliga lösningar utifrån förbestämda krav. Samtidigt implementerades ett system för att testa tekniken JSON Scheman. Beslutet togs att befintliga verktyg skulle delvis eller helt användas om det var möjligt utifrån de förbestämda kraven.

\section{Avgränsningar}
\label{sec:intro:avgränsningar}
En viktig avgränsning är att rapporten endast väldigt ytligt undersöker olika grafiska användargränssnitt, och användbarheten hos dem. Arbetet handlar inte primärt om användbarhet, utan arbetet handlar i större grad om hur JSON Schema kan automatisera skapandet av användbara gränssnitt. Med hjälp av kunskapen som arbetet presenterar kan användbara grafiska användargränssnitt enklare skapas.

Ett annat ämne som också är viktigt är säkerhet av systemet som skapas. Säkerheten hos applikationen omfattas inte av arbetet, men det ignoreras samtidigt inte. Systemet som utvecklas för att utbyta JSON Scheman och JSON-data sker över en SSL-krypterad säker uppkoppling. Det här arbetet utvärderar inte säkerheten hos den uppkopplingen.

Att skapa ett användbart grafiskt användargränssnitt utifrån alla möjliga sorters JSON Scheman med samma verktyg omfattas inte av arbetet. Arbetet kommer utforska olika strategier och metoder för att arbeta med förutbestämda JSON Scheman. Dessutom kommer bara en delmängd av JSON Scheman att hanteras, och en färdig JSON Schema parser skapas och testas ej.

Validering av data är något som webbtjänster ofta måste ta hänsyn till. En klient kan annars skicka otillåten data till en webbserver och därför måste webbservern alltid validera data när den tar emot data, innan data används eller lagras. Ett stort användningsområde av JSON Schema är validering men det omfattas inte av arbetet. Valideringsdata i JSON Schema kommer endast användas för att extrapolera annoteringsinformation. Anledningen till att validering inte omfattas är att inmatningsdata valideras hos klienten, samt att både klienten och servern omfattas av arbetet. En till faktor är att användarna inte anses ha uppsåt att förstöra eller falsifiera data.

\section{Disposition}
\label{sec:intro:disposition}
Kapitel \ref{sec:teori} presenterar den teoretiska bakgrunden. Kapitel \ref{sec:metod} beskriver metodologin som valdes. Kapitel \ref{sec:forarbete} presenterar och utvärderade kända implementationer som berörde liknande ämnesområden som det här arbetet. Kapitel \ref{sec:arbetet} redovisar hur systemet utvecklades, vilka val som togs, varför de valen togs samt diskuterar hur arbetet förhåller sig till tidigare implementationer av liknande projekt och verktyg. Kapitel \ref{sec:slutsats} diskuterar hur bra JSON Schema är som verktyg för att annotera manipulerbar data, vilka outforskade intresseområden som skulle kunna behöva fortsatt arbete, samt utvärderar arbetets resultat.
\chapter{JSON och JSON Scheman}
\label{sec:teori}
Det här kapitlet beskriver vad JSON och JSON Scheman är, samt hur de används. Kapitel \ref{sec:teori:json} beskriver vad JSON är. Kapitel \ref{sec:teori:json-web} beskriver hur JSON används för kommunikation mellan webbtjänster. Kapitel \ref{sec:teori:schema} beskriver JSON Scheman, vad de är och hur de är specificerade. Kapitel \ref{sec:teori:schema-användningsområden} diskuterar användningsområden av JSON Schema samt listar kända implementationer.

\section{JSON}
\label{sec:teori:json}
JSON erbjuder stöd för några enkla datatyper: textsträngar,  siffror, tomt värde samt booleska värden, som presenteras i figur \ref{tab:json-primitives}. JSON erbjuder dessutom stöd för två komplexa datatyper vilket är vektorer \textit{(array)}, en ordnad lista av JSON-värden, samt objekt \textit{(object)}, vilket är en oordnad mängd av namn-värde-par \textit{(properties)}. Exempel på de två komplexa datatyperna visas i figur \ref{fig:json-komplex-example}. Resten av rapporten kommer utbytbart använda JSON-värde, JSON-data, JSON-fil och JSON-dokument för att förklara en av de sex datatyperna som kan representeras med JSON.

\begin{table}
	\centering
	\caption{De primitiva datatyperna i JSON}
	\label{tab:json-primitives}
	\begin{tabular}{ | l | l | p{2.2cm} | }
		\hline
		Datatyp & Namn i JSON och JavaScript & Exempel \\
		\hline
		Textsträng & String & \mintinline{json}{"hej värld"} \\
		\hline
		Siffra & Number & \mintinline{json}{4} \\
		\hline
		Tomt värde & Null & \mintinline{json}{null} \\
		\hline
		Booleskt värde & Boolean & \mintinline{json}{false} \\
		\hline
	\end{tabular}
\end{table}

\begin{figure}
	\begin{subfigure}[t]{0.47\textwidth}
		\begin{minted}[tabsize=2, frame=single, fontsize=\small, framesep=2mm]{json}
[
	"hej värld",
	4,
	null,
	false
]
		\end{minted}
		\vspace{-1.2em}
		\caption{Exempel på JSON-array}
		\label{fig:json-array-example}
	\end{subfigure}\hfill
	\begin{subfigure}[t]{0.47\textwidth}
		\begin{minted}[tabsize=2, frame=single, fontsize=\small, framesep=2mm]{json}
{
	"firstName": "Erik",
	"lastName": "Andersson",
	"age": 30,
	"human": true
}
		\end{minted}
		\vspace{-1.2em}
		\caption{Exempel på JSON-object}
		\label{fig:json-object-example}
	\end{subfigure}
	\caption{De komplexa datatyperna i JSON}
	\label{fig:json-komplex-example}
\end{figure}

\noindent
Med hjälp av att rekursivt använda \textit{array} eller \textit{object} går det att representera komplexa datastrukturer med hjälp av JSON. Det finns inga begränsningar i hur komplext datastrukturer kan representeras.

\section{JSON i webbkommunikation}
\label{sec:teori:json-web}
På grund av att JSON är kompakt, enkelt läsbart och har brett stöd hos många språk och implementationer, har JSON blivit väldigt utbrett bland webbtjänster. Figur \ref{fig:json-exchange-example} visar ett exempel på en hypotetiskt transaktion av data på webben. En hypotetisk förfrågan till en webbtjänst skulle kunna se ut som i Figur \ref{fig:json-request-example}, där en klient förfrågar om de nuvarande väderförhållandena i Stockholm i Sverige. Svaret från webbservern skulle kunna se ut som i Figur \ref{fig:json-response-example} där webbservern svarar att temperaturen är minus tre grader Celsius och att det snöar. Exemplet visar hur simpelt JSON som dataformat är att förstå, vilket delvis skulle kunna vara en förklaring för populariteten.

\begin{figure}
	\begin{subfigure}[t]{0.47\textwidth}
		\begin{minted}[tabsize=2, frame=single, fontsize=\small, framesep=2mm]{json}
{
	"country": "Sweden",
	"city": "Stockholm"
}
		\end{minted}
		\vspace{-1.2em}
		\caption{Exempel på förfrågan till webbserver}
		\label{fig:json-request-example}
	\end{subfigure}\hfill
	\begin{subfigure}[t]{0.47\textwidth}
		\begin{minted}[tabsize=2, frame=single, fontsize=\small, framesep=2mm]{json}
{
	"timestamp": "06/01/2018 10:45:08",
	"country": "Sweden",
	"city": "Stockholm",
	"weather": "Snowing",
	"temperature": -3
}
		\end{minted}
		\vspace{-1.2em}
		\caption{Exempel på svar på förfrågan från webbserver}
		\label{fig:json-response-example}
	\end{subfigure}
	\caption{Exempel på datatransaktion mellan webbklient och webbserver}
	\label{fig:json-exchange-example}
\end{figure}

\section{JSON Schema}
\label{sec:teori:schema}
JSON Schema är ett ramverk för att förklara hur JSON-värden kan se ut. JSON Schema specificerar regler som kan användas för att antingen bestämma om befintliga JSON värden är giltiga, eller för att i förväg beskriva hur giltiga värden får se ut. Objektet i figur \ref{fig:json-object-example} skulle kunna valideras enligt JSON Schemat i figur \ref{fig:json-schema-example}. Den senaste fastslagna versionen \textit{(Draft 7)} av ramverket bygger på tre dokument: \textit{Core}, \textit{Validation} samt \textit{Hyper-Schema}. \cite{A.Wright,Andrews2018,Andrews2018a}

\begin{figure}
	\begin{minted}[tabsize=2, frame=single, fontsize=\small, framesep=2mm]{json}
{
	"type": "object",
	"required": ["firstName", "age"],
	"properties": {
		"firstName": { "type": "string" },
		"lastName": { "type": "string" },
		"age": { "type": "integer" },
		"human": { "type": "boolean" }
	}
}
	\end{minted}
	\vspace{-1.7em}
	\caption{Exempel på simpelt JSON Schema}
	\label{fig:json-schema-example}
\end{figure}

\subsection{JSON Schema Core}
JSON Schema Core täcker grunderna för JSON Schema. Dokumentet fastställer exempelvis mediatypen som borde användas för att skicka JSON Scheman över HTTP, förhållandet mellan flera JSON Scheman, samt hur heltal borde behandlas. Att JSON Scheman själva är JSON-dokument bestäms också. Dokumentet fastställer också att validering och annotering av JSON-värden ska ske enligt dokumentet draft-handrews-json-schema-validation-01 \textit{(Validation)}, samt att draft-handrews-json-schema-hyperschema-01 \textit{(Hyper-Schema)} behandlar reglerna kring att beskriva hypertextstrukturen hos JSON-dokument. \cite{A.Wright}

\subsection{JSON Schema Validation}
JSON Schema Validation beskriver tre saker: hur man beskriver ett JSON-dokument, hur man ger tips åt användargränsnitt för att jobba med JSON-dokument samt hur man kan beskriva påståenden om ett dokuments validitet. Förenklat beskriver det här dokumentet strukturen hos ett JSON Schema, med beskrivningar av nästan alla nyckelorden. Utöver att beskriva hur JSON-dokument ska valideras, presenteras annoteringsnyckelord som \mintinline{json}{"title"} och \mintinline{json}{"description"}, där \mintinline{json}{"title"} är en kort förklaring för JSON värdet den validerar, och \mintinline{json}{"description"} är en längre förklaring. \cite{Andrews2018}

\subsection{JSON Schema Hyper-Schema}
JSON Schema utvecklas till stor del för användandet av JSON Scheman i webbtjänster. Därför beskriver det tredje dokumentet, JSON Schema Hyper-Schema, hur resurser kan manipuleras och interageras med över hypermediamiljöer som HTTP. JSON Schema Validation skulle kunna beskriva hur ett API anrop ska hanteras och vad som förväntas från förfrågningar och svar på dem. JSON Schema Hyper-Schema kan då användas för att beskriva ett helt API och hur de olika anropen och resurserna är relaterade till varandra. \cite{Andrews2018a}

\subsection{Kontroversiella flyttal i JSON Schema}
\label{sec:teori:schema:float}

JSON Schema stöder två nyckelord för siffror: \textit{number}, och \textit{integer}, vilket motsvarar siffror respektive heltal \cite{Andrews}. Nyckelordet \textit{number} betyder inte nödvändigtvis flyttal då JSON och JavaScript inte skiljer på heltal och flyttal, vilket en betydligt stor andel programmeringsspråk gör som Python, Ruby, C, C++, C\#, Java, Delphi med många fler \cite{Embarcadero,Oracle,Microsofta,GNU,GNUa,Britt,Britta,PythonSoftwareFoundation2018,ECMA2013,EcmaInternational2017}. JSON Schema definerar ett heltal som alla siffror med en bråkdel som är lika med noll \cite{Andrews}. Det skulle betyda att både talet 1 och 1.0 skulle tolkas som ett heltal. Många JSON parsers i många språk tolkar 1.0 som ett flyttal vilket gör det svårt att kontrollera om en siffra är ett heltal. Flera implementationer av JSON Schema parsers som exempelvis Python-baserade jsonschema tolkar 1.0 som ett flyttal, trots att JSON Schema specificerar motsatsen \cite{SpaceTelescopeScienceInstitute2016}.

Det finns andra oklarheter kring flyttal, som att det är svårt att validera om ett flyttal är en multipel av ett annat tal, då få språk erbjuder exakt nogranhet för flyttal \cite{Cederqvist2017}. Vissa föreslår att flyttal borde hanteras som textsträngar med nyckelordet \textit{format}, men då krävs en större bredd av valideringstermer för att erbjuda samma funktionalitet som det redan finns till datatyper av typen \textit{number} \cite{Poberezkin,Faassen}.

\section{Användningsområden för JSON Schema}
\label{sec:teori:schema-användningsområden}
Användningsområden för JSON Scheman är bland annat:

\begin{enumerate}
	\item Validering av data.
	\item Annotering av data.
	\item Beskrivning av REST APIer.
	\item Automatisk generering av kompatibel kod, för att hantera JSON värden beskrivna med JSON Schema.
	\item Automatisk generering av API-dokumentation.
	\item Automatisk generering av användargränsnitt.
\end{enumerate}

\noindent
Att använda JSON Schema för användningsområdena 1-3 är trivialt. Det går att utveckla program som kan hantera alla oändligt möjliga permutationer av JSON Schema. Det som däremot inte är trivialt är hur användningsområdena 4-6 skulle kunna generaliseras så pass mycket att ett program eller algoritm skulle kunna hantera vilket giltigt JSON Schema som helst. Användningsområde fyra och fem omfattas inte av den här rapporten, och varför användningsområde sex inte är trivialt diskuteras mer i resultatet.
% Byt ut resultatet mot rätt kapitel

The Json Schema organisation listar kända implementationer på sin hemsida, och har delat upp dem i följande kategorier \cite{TheJSONSchemaorganisation}:
\begin{itemize}
	\item Validators
	\item Hyper-Schema
	\item Schema generation
	\item Data parsing
	\item UI generation
	\item Editors
	\item Compatibility
	\item Documentation generation
\end{itemize}
\noindent
Arbetet kommer behöva implementera tre av de listade implementationerna: Schema generation, Data parsing samt UI generation, vilket diskuteras i kapitel \ref{sec:teori:schema-användningsområden:generering}, \ref{sec:teori:schema-användningsområden:parsning} samt \ref{sec:teori:schema-användningsområden:ui-generering}.

\subsection{Generering av scheman}
\label{sec:teori:schema-användningsområden:generering}
Schemagenerering som kategori består av tolv implementationer där det går att ytterligare dela upp implementationerna i tre kategorier. Det finns implementationer som utgår från JSON data, och genererar ett JSON Schema för att beskriva datan. Det kan användas om det går att anta att all användning av JSON Schemat kommer att användas på data med exakt likadan struktur. Den andra kategorin av implementationer är implementationer som genererar JSON Scheman utifrån kända datatyper i ett statiskt typat språk. Den tredje kategorin av implementation är implementationer som erbjuder en annan metod att beskriva datan, för att sedan översätta det till ett JSON Schema. \cite{TheJSONSchemaorganisation}

\subsubsection{Implementationerna som genererar JSON Scheman från JSON data:}
\begin{itemize}
	\item Schema Guru \textit{(Scala)} \cite{Snowplow}
	\item JSON Schema Generator \textit{(Visual Studio)} \cite{MadsKristensen}
	\item json-schema-generator \textit{(JavaScript / JSON)} \cite{Romanovich}
\end{itemize}

\subsubsection{Implementationerna som genererar JSON Scheman från statiska datatyper inbyggda i språket:}
\begin{itemize}
	\item Json.NET Schema \textit{.NET} \cite{Newtonsoft}
	\item NJsonSchema for .NET \textit{.NET} \cite{Suter}
	\item typescript-json-schema \textit{(TypeScript)} \cite{El-Dardiry}
	\item Typson \textit{(TypeScript)} \cite{Bovet}
\end{itemize}

\subsubsection{Implementationerna som genererar JSON Scheman från andra liknande beskrivningar:}
\begin{itemize}
	\item Liform \textit{(PHP)} \cite{Limenius}
	\item JSL \textit{(Python)} \cite{Romanovich}
	\item JSONSchema.net \textit{(Online webbverktyg)} \cite{Bovet}
	\item Schema Guru Web UI \textbf{Obs:} Verktyget hittades ej och kommer därför exkluderas från resten av rapporten.
	\item APIAddIn \textit{(Sparx Enterprise Architect)} \cite{Tomlinson}
\end{itemize}

\subsection{Parsning av JSON Scheman}
\label{sec:teori:schema-användningsområden:parsning}
En parser tolkar JSON Scheman och representerar schemat med någon annan datastruktur. Ofta är parsning viktigt för att schemat ska kunna representeras med en datastruktur som programmeringsspråket är kompatibelt med. Vissa implementationer använder ett färdigt JSON Schema och genererar kod som är kompatibelt med att hantera JSON som är formaterad utifrån schemats struktur. Andra implementationer kan dynamiskt hantera vilket schema som helst under exekvering, och dynamiskt skapa parsers för JSON formaterad utifrån schemat. De parsers som listas på The Json Schema organisations hemsida är följande:

\begin{itemize}
	\item DJsonSchema \textit{Delphi} \cite{Schlothauer&WauerGmbH}
	\item jsonCodeGen \textit{Groovy} \cite{Schlothauer&WauerGmbHa}
	\item aeson-schema \textit{Haskell} \cite{Kowalczyk}
	\item AutoParse \textit{Ruby} \cite{Googleb}
	\item json-schema-codegen \textit{Scala} \cite{Tundra}
	\item Argus \textit{Scala} \cite{Fenton}
	\item Bric-à-brac \textit{Swift} \cite{GlimpseI/OInc}
	\item gojsonschema \textit{Golang} \textbf{Obs:} Verktyget saknade information på engelska eller svenska och kommer därför exkluderas från resten av rapporten. \cite{Zhangtao}
	\item jsonschema \textit{Golang} \cite{Qriinc.}
\end{itemize}

\subsection{Tidigare försök av generering av användargränsnitt baserade på JSON Scheman}
\label{sec:teori:schema-användningsområden:ui-generering}
Det finns olika implementationer av att generera ett användargränsnitt utifrån ett JSON Schema. Samtliga kända implementationer är skrivna i språket JavaScript och bemöter därför ingen av svårigheterna med att använda JSON eller JSON Schema med andra språk. Samtliga implementationer är implementationer för att generera hemsidor eller komponenter till hemsidor, vilket skiljer sig mycket mot att generera användargränsnitt åt Windows med Delphi, vilket arbetet gjorde.

Vissa av implementationerna används för att generera ett användargränsnitt för att förklara ett API beskrivet med JSON Schema och andra implementationer används för att generera ett formulär för att manipulera data beskrivet av JSON Schema. Att generera ett formulär för att manipulera data beskrivet av JSON Schema är exakt vad den här rapporten utvärderar. Användargränsnittsgenererarna som listas på The Json Schema organisations hemsida är följande:

\begin{itemize}
	\item Alpaca Forms \cite{GitanaSoftwareInc.}
	\item Angular Schema Form \cite{Textalk}
	\item Angular2 Schema Form \cite{MakinaCorpus}
	\item JSON Editor \cite{JeremyDorn}
	\item JSON Form \cite{Joshfire}
	\item json-forms \cite{Brutusin.org}
	\item JSONForms  \cite{EclipseSource}
	\item Jsonary  \textbf{OBS!}
	\item liform-react \cite{NachoMartin}
	\item Metawidget \cite{Metawidget}
	\item pure-form webcomponent \textbf{Obs not found}
	\item React JSON Schema Form \cite{MozillaServices}
	\item React Schema Form \cite{NetworkNewTechnologiesInc.}
\end{itemize}

%\section{JSON Schema förklarat med BNF}
%
%
%\inputminted[tabsize=2, frame=single, fontsize=\small, framesep=2mm]{bnf}{code/json-schema.bnf}
%\inputminted[tabsize=2, frame=single, fontsize=\small, framesep=2mm]{ebnf}{code/json-schema.ebnf}
%\inputminted[tabsize=2, frame=single, fontsize=\small, framesep=2mm]{ebnf}{code/strSch.ebnf}


%	<statement> ::= <ident> `=' <expr> 
%	\alt `for' <ident> `=' <expr> `to' <expr> `do' <statement> 
%	\alt `{' <stat-list> `}' 
%	\alt <empty> 
%
%	<stat-list> ::= <statement> `;' <stat-list> | <statement> 
%
%	<statement> ::= <ident> `=' <expr> 
%	\alt `for' <ident> `=' <expr> `to' <expr> `do' <statement> 
%	\alt `{' <stat-list> `}' 
%	\alt <empty> 
%	
%	<stat-list> ::= <statement> `;' <stat-list> | <statement> 



%\begin{syntdiag}
%	<ident> ‘(’
%	\begin{rep} \begin{stack} \\
%			<type> \begin{stack} \\ <ident> \end{stack}
%		\end{stack} \\ ‘,’ \end{rep}
%	\begin{stack} \\ ‘...’ \end{stack} ‘)’
%\end{syntdiag}


\chapter{Metodologi}
\label{sec:metod}

Metodologin kan delas upp i tre delar. Först utvärderades 29 olika implementationer som erbjuder liknande funktionalitet som delar av systemet som skulle utvecklas. Sedan utvecklades systemet. Där utvärderades olika alternativ på lösningar och JSON Schema som verktyg utvärderades. Både hur JSON Schema kan användas och vad det erbjuder men också vad JSON Schema inte erbjuder och hur det kan lösas. Sist utvärderades arbetet och slutsatser drogs kring JSON Schema som verktyg.

\section{Systemet att förhålla sig till}
Det grafiska användargränssnittet som skulle utvecklas för att manipulera data på operatörsdatorer, skulle vara en del av administratörsprogrammet som bestod av flera delar. Dessutom skulle servern på operatörsdatorerna bli ett tillägg till befintliga Mimer. Därför behövde arbetet av teknisk anledning ta hänsyn till hur LSE utvecklade resten av systemet med val av programmeringsspråk och utvecklingsmiljö. LSE arbetar huvudsakligen med språket Delphi för att utveckla skrivbordsapplikationer till Windows, vilket båda komponenterna av systemet är. Därför utvecklades arbetet i programmeringsspråket Delphi.

\section{Utvärdering av kända verktyg}

För att skapa en medvetenhet över hur andra tidigare löst eller försökt lösa liknande problem utvärderades och testades 29 olika implementationer som erbjuder liknande funktionalitet. Det skapade en bra bild över vad som krävdes för att genomföra arbetet. Utvärderingen beskrivs i kapitel \ref{sec:forarbete} och hur implementationerna förhåller sig till det praktiska arbetet som genomfördes beskrivs i kapitel \ref{sec:arbetet}.

\section{Skapande av systemet}
Efter utvärderingen konstruerades systemarkitekturen för systemet, så att resten av arbetet kunde planeras utifrån det. De stora praktiska problemenen som skulle lösas blev:

\begin{itemize}
	\item Hur ska användarinteraktion integreras i detta system, så att interaktion faktiskt manipulerar data?
	\item Hur ska schemat som formuläret baseras på, genereras?
	\item Hur ska schemat parsas, tolkas och sedan representeras i språket Delphi?
	\item Hur ska schemats representation tillsammans med data presenteras i ett grafiskt användargränssnitt?
\end{itemize}

De fyra frågorna diskuteras och besvaras i varsin del av kapitel \ref{sec:arbetet}.

\section{Utvärdering av arbetet}
Efter att arbetet blev genomfört och färdigt utvärderades det. Det utvärderades bland annat utifrån hur generellt systemet blev. Förbättringar på systemet diskuteras och fortsatt arbete föreslås, inte bara för systemet i arbetet utan primärt fortsatta utvecklingsområden hos JSON Schema.
\chapter{Utvärdering av befintliga verktyg}
\label{sec:forarbete}
För att förstå hur systemet skulle utvecklas, och om det redan fanns liknande lösningar utvärderades alla kända implementationer som berörde liknande ämnesområden som det här arbetet. Implementationerna presenterades i kapitel \ref{sec:teori:schema-användningsområden} och kommer diskuteras mer djupgående i detta kapitel. Resultatet från utvärderingen av verktyg som genererar eller parsar JSON Scheman var att inget verktyg passade användningsområdet, och att en implementation var trivial. Därför beslutades det att både generering och parsning implementeras skulle skapas av författaren. Samtliga verktyg som genererar grafiska användargränssnitt utifrån JSON Scheman var anpassade för hemsidor \textit{(HTML, CSS och JavaScript)}, vilket hindrade verktygen från att integreras med systemet som skrevs i Delphi som en skrivbordapplikation till Windows. Därför fick även det verktyget skapas av författaren.

\section{Implementationer som utesluts ur rapporten}
Vissa av implementationerna uteslöts från rapporten då det ansågs vara omöjligt att förstå implementationerna. \textit{Schema Guru Web UI} kunde inte hittas. \textit{AutoParse} är ett verktyg som inte uppdaterats sedan 26 Mars 2013, vilket betyder att den som bäst kan implementera version tre av JSON Schema \cite{Googleb}. Dessutom länkar projektet till en hemsida som länkar tillbaka till projektet, vilket saknar dokumentation \cite{Googleb}. Verktyget \textit{gojsonschema} saknade tillräcklig information på engelska eller svenska \cite{Zhangtao}. \textit{Jsonary} hade bristande dokumentation, med länkar till en hemsida som inte finns \cite{Jsonary-js}. Länken till \textit{pure-form webcomponent} på Json Schemas hemsida returnerade felkoden 404 vilket betecknar att webbsidan som efterfrågats inte finns eller inte kan hittas. Efter omfattande sökningar hittades inte verktyget.

\section{Generering av scheman från JSON}
\label{sec:forarbete:json-till-schema}
Att generera JSON Scheman från en JSON-fil perfekt, är omöjligt. Att bara observera en JSON-fil tillför inte tillräcklig infomation för att generera ett JSON Schema. Studera exemplet i figur \ref{fig:json-super-simple-example}. Att skapa ett JSON Schema som beskriver det objektet skulle kunna se ut som i figur \ref{fig:schema-super-simple-example-1}. Då har antaganden tagits om att det objektet alltid ska ha en \textit{property} som heter \textit{name} och ska innehålla en textsträng. Det skulle kunna vara så att \textit{name} alltid ska innehålla en textsträng som börjar på stor bokstav, vilket är rimligt när det representerar ett namn, och då skulle schemat se ut som i figur \ref{fig:schema-super-simple-example-2}. Det skulle annars kunna vara så att \textit{name} inte får vara vilken textsträng som helst utan måste vara en av två textsträngar. Det skulle till och med kunna vara så att \textit{name} också skulle kunna vara det booleska värdet \textit{false}. Då skulle schemat se ut som i figur \ref{fig:schema-super-simple-example-3}.

\begin{figure}
	\inputminted[tabsize=2, frame=single, fontsize=\small, framesep=2mm]{json}{code/schema-generation-example/json-file.json}
	\vspace{-1.7em}
	\caption{Exempel på enkelt JSON-objekt}
	\label{fig:json-super-simple-example}
\end{figure}

Alla de här exemplen har enbart behandlat feltolkning av en \textit{property} och inte diskuterat ännu större missförstånd med objektets struktur. Objektet skulle möjligtvis kunna ha en valfri \textit{property} som representerar efternamn. Då skulle schemat se ut som i figur \ref{fig:schema-super-simple-example-4}. Alla exempel är giltiga scheman till objektet i det givna exemplet. Det finns ingen möjlighet att förstå vilket schema som faktiskt beskriver datan, utan att göra många antaganden om datan. Därför ansågs samtliga lösningar som genererar JSON Scheman från JSON-filer vara otillräckliga för projektet.

\begin{figure}
	\begin{subfigure}[t]{0.47\textwidth}
		\inputminted[tabsize=2, frame=single, fontsize=\small, framesep=2mm]{json}{code/schema-generation-example/schema-example1.json}
		\vspace{-1.2em}
		\caption{Exempel på genererat schema}
		\label{fig:schema-super-simple-example-1}
	\end{subfigure}\hfill
	\begin{subfigure}[t]{0.47\textwidth}
		\inputminted[tabsize=2, frame=single, fontsize=\small, framesep=2mm]{json}{code/schema-generation-example/schema-example2.json}
		\vspace{-1.2em}
		\caption{Exempel på genererat schema med strängmönster}
		\label{fig:schema-super-simple-example-2}
		\vspace{.8em}
	\end{subfigure}
	\begin{subfigure}[t]{0.47\textwidth}
		\inputminted[tabsize=2, frame=single, fontsize=\small, framesep=2mm]{json}{code/schema-generation-example/schema-example3.json}
		\vspace{-1.2em}
		\caption{Exempel på genererat schema med \textit{enums}}
		\label{fig:schema-super-simple-example-3}
	\end{subfigure}\hfill
	\begin{subfigure}[t]{0.47\textwidth}
		\inputminted[tabsize=2, frame=single, fontsize=\small, framesep=2mm]{json}{code/schema-generation-example/schema-example4.json}
		\vspace{-1.2em}
		\caption{Exempel på genererat schema med dolda \textit{properties}}
		\label{fig:schema-super-simple-example-4}
	\end{subfigure}
	\caption{Olika JSON Scheman som är kompatibla med JSON-objektet i figur \ref{fig:json-super-simple-example}}
	\label{fig:schema-super-simple-example-group}
\end{figure}

\section{Generering av scheman från statiska datatyper i statiskt typade programmeringsspråk}
Det finns implementationer som utnyttjar att statiskt typade programmeringsspråk redan innehåller beskrivningar av data som ska bearbetas. Både .Net och TypeScript erbjuder programmeringsfunktioner inbyggda i språket för att beskriva komplexa datastrukturer. Att sedan översätta dem till JSON Schema-filer presterar riktigt bra. Varken .Net eller TypeScript erbjuder stöd för att exempelvis bestämma att ett tal bara får befinna sig inom ett bestämt intervall, och det saknas fler specificiteter som JSON Schema erbjuder. För att bemöta de bristerna använder samtliga implementationer speciellt formaterade kommentarer som kallas annotationer, vilket möjliggör all funktionalitet som JSON Schema erbjuder. Att TypeScript är ett språk som kompileras till JavaScript, vilket JSON och i sin tur JSON Scheman bygger på innebär att översättningen mellan datatyper i TypeScript och JSON Scheman resulterar i väldigt korrekta resultat. \cite{Newtonsoft,Suter,El-Dardiry,Bovet}

Implementationer skrivna för .Net eller TypeScript ansågs vara svåra och möjligtvis omöjliga att integrera i systemet. Det är självklart en subjektiv bedömning tagen av författaren. Bedömningen baseras på att all kod som hanterar JSON Scheman måste vara skrivet i antingen plattformen .Net eller språket TypeScript. Det skulle innebära att en stor del av systemet vore skrivet i ett språk som inte är Delphi, samt att de två delarna av systemet skulle behöva integreras. Då generering av JSON Schema är relativt trivialt ansågs det vara enklare och mer långsiktigt att skapa ett eget verktyg.

\section{Andra genererare av scheman}
\textit{Liform} och \textit{JSL} erbjuder metoder och funktioner som underlättar dynamiskt skapande av JSON Scheman under exekvering. De erbjuder datatyper som är inbyggda i språket för att definiera och hantera komponenter av JSON Scheman. De kan sedan generera ett JSON Schema utifrån de här instanserna av datatyperna. \cite{Romanovich,Limenius}

\textit{Liform} ansågs vara väldigt enkelt att använda och passade användningsområdet. Det använder dessutom den senaste standarden av JSON Schema. Däremot är \textit{Liform} skrivet i och för språket PHP, vilket inte är kompatibelt med Delphi. Därför ansågs inte \textit{Liform} vara enkelt att integrera med resten av systemet.

\textit{JSL} ansågs också passa användningsområdet men använde inte den senaste standarden av JSON Schema. Dessutom var \textit{JSL} skrivet i och för språket Python vilket inte är kompatibelt med Delphi.

\textit{APIAddIn} är ett plugin åt \textit{Sparx Enterprise Architect}, vilket är ett verktyg för modellering, visualisering och design av system, mjukvara, processer eller arkitekturer. Verktyget baseras på UML vilket är ett generellt språk för modellering av system. Det erbjuder delvis en liknande funktion som JSON Schema erbjuder. Om datan som skulle beskrivas av JSON Scheman redan fanns definierade med UML, skulle det möjligtvis vara ett rimligt alternativ att överväga, men då datan inte fanns definierat med UML ansågs APIAddIn vara ett onödigt verktyg, när schemat lika gärna kan definieras med JSON Schema från första början. \cite{Tomlinson}

Online-verktyget \textit{JSONSchema.net} är ett verktyg för att skapa JSON Scheman, med hjälp av ett grafiskt användargränssnitt. Det erbjöd inte mer funktionalitet än att skriva JSON Schemat manuellt. Dessutom genererade den inte JSON Scheman enligt den senaste standarden. \cite{Jackwootton}

\section{Parsning av JSON Scheman}

En viktig del i all hantering av JSON Scheman är självklart parsningen, vilket är utförandet av att läsa och tolka JSON Schemat, för att sedan representera innehållet på nytt med en annan representation som är mer användbar för syftet. Vissa parsers tolkar ett JSON Schema, och genererar kod för att hantera JSON-filer som är strukturerade efter det schemat. Det kan också användas för att programmet dynamiskt ska förstå hur en JSON-fil ska läsas. Dessutom kan det handla om att parsa ett JSON Schema för att skapa ett dynamiskt test för att testa om given JSON-data är korrekt formaterad, utifrån det givna schemat.

\textit{DJsonSchema} är ett verktyg som parsar ett JSON Schema och sedan genererar kod som klarar av att parsa JSON som följer samma struktur som schemat. DJsonSchema är skrivet i Delphi och genererar kod för Delphi, vilket är samma språk som konfigurationssystemet skrevs i. Systemet krävde dynamisk parsning av scheman så därför passar inte DJsonSchema för parsning i detta projekt. Utöver det saknades stöd för version sju av JSON Schema. DJsonSchema uppger dessutom att implementationen var ofullständig. \cite{Schlothauer&WauerGmbH}

\textit{jsonCodeGen} hade bristfällig dokumentation. Det framstod att det var ett verktyg som parsar en utökad version av JSON Scheman som inte var kompatibel med den senaste officiella, eller någon tidigare officiell JSON Schema specifikation. Dessutom var verktyget skrivet för Groovy, vilket också gör den kompatibel med Java, men inte Delphi. Utifrån beskrivningen av hur verktyget skulle användas framstod det som att verktyget genererar statiska filer utifrån JSON Scheman eller JSON-filer, vilket inte är dynamiskt som konfigurationssystemet kräver. Det är också värt att nämna att utvecklarna av jsonCodeGen är samma utvecklare som utvecklade DJsonSchema. \cite{Schlothauer&WauerGmbHa} 

\textit{aeson-schema} är ett verktyg för att validera ett JSON-värde mot ett schema, eller för att generera en JSON parser åt JSON-filer som är strukturerade efter ett givet schema. Det garanterar att om ett JSON-värde blivit godkänd i validering, kan samma JSON-värde parsas och användas i ett program skrivet i Haskell. Verktyget implementerar version tre av JSON Schema, vilket inte är version sju som projektet använder. \cite{Kowalczyk}

\textit{json-schema-codegen} stöder stora delar av JSON Schema men inte allt. Verktyget parsar ett JSON Schema och genererar sedan kod som kan parsa JSON som är strukturerade efter schemat. \cite{Tundra}

\textit{Argus} är ett verktyg som erbjuder nästan exakt den funktionalitet arbetet skulle implementera, vilket var att dynamiskt parsa JSON Scheman och sedan erbjuda en JSON parser som parsar JSON-filer som har strukturen som är beskriven av schemat. Parsern som parsar JSON-filerna kunde sedan erbjuda representationer av datan i form av datastrukturer som är kompatibla med programmeringsspråket programmet är skrivet i. Argus implementerades i Scala, vilket inte är kompatibelt med Delphi. \cite{Fenton} \textit{Bric-à-brac} är också ett verktyg som liknade Argus i funktionalitet, fast implementerat i språket Swift \cite{GlimpseI/OInc}. En notering är att dokumentationen är motstridig då den påstår att verktyget erbjuder en oföränderlig \textit{immutable} datastruktur, men den i exempelkoden visar motsatsen \cite{GlimpseI/OInc}. Det skulle kunna vara en indikator för att projektet inte är helt felfritt.

\textit{jsonschema} är ett verktyg som dynamiskt parsar JSON Scheman och erbjuder en instans som kan användas för att validera JSON-filer. Projektet är skrivet i Golang, vilket inte är kompatibelt med Delphi. \cite{Qriinc.}

Många av dessa projekt saknar tillräcklig dokumentation, vilket gör det omöjligt att använda dem i ett projekt. Dessutom är parsningen av JSON Scheman relativt enkelt, då det enbart handlar om att parsa JSON-filer som följer en förutbestämd struktur. Dessutom skiljer sig användningsområdena sig starkt efteråt, vilket kan kräva unik anpassning av JSON Schema-parsern, vilket är en stor anledning till att det finns så många olika sorters parsers. Ett annat problem att ta hänsyn till med implementationerna är att många av dem bara erbjuder generering av statisk kod i förväg, och klarar inte av att tolka olika JSON Scheman under exekvering av programmet. Det här arbetet handlar om att dynamiskt tolka olika sorters scheman för att anpassa ett användargränssnitt till servern som klienten kommunicerar mot, vilket innebär att de implementationerna är oanvändbara. Det sista och största problemet med implementationerna är att bara en var kompatibel med Delphi, vilket är språket som systemet skrevs i, och det ansågs vara otillräckligt. Med hänvisning till att det är relativs enkelt att skriva en parser och med problematiken alla implementationer presenterade, skrevs en egen parser åt det här arbetet.

\section{Användargränssnitt genererade baserat på JSON Scheman}
\label{sec:forarbete:gui-generering}
Samtliga implementationer som genererar grafiska användargränssnitt utifrån JSON Scheman, gör det för hemsidor \textit{(HTML, CSS och JavaScript)}. Att generera ett grafiskt användargränssnitt, baserat på JSON Scheman, med ett annat språk än JavaScript kan presentera hinder eller svårigheter. Att använda ett webbaserat grafiskt användargränssnitt var inte ett alternativ för systemet men samtliga implementationer undersöktes, för att ta lärdom om hur de fungerade. Många lösningar separerar JSON Schemat med datan som JSON Schemat beskriver. Den faktiska datan som det grafiska användargränssnittet visar, och sedan manipulerar kommer också kallas modell i resten av rapporten.

Att använda JSON Schema för att generera grafiska användargränssnitt kan ske på olika sätt, och många implementationer bygger på att ytterligare information tillförs för att kunna specificera hur modellerna ska representeras i det grafiska användargränssnittet. Vissa lösningar är generella för användning på vilken hemsida som helst, medan andra är kopplade till att användas med ett visst ramverk, som React, JQuery eller Angular.

Alpaca Forms är en unik lösning då den är den enda implementationen som inte använder ett JSON Schema som är formaterat med JSON, utan schemat är en instans av ett JavaScript-objekt, vilket möjliggör att schemat inkluderar funktionslogik; En funktionalitet som saknas hos JSON. Utöver funktionslogik, inkluderar schemat en \textit{property} som kallas \textit{options}, som används för att beskriva hur det grafiska användargränssnittet ska se ut. De delarna av schemat som är korrekt JSON Schema används till stor del enbart för att koppla ihop det grafiska användargränssnittet med datan. \cite{GitanaSoftwareInc.}

Andra lösningar som utökade det officiella JSON Schemat är:
\begin{itemize}
	\item Angular2 Schema Form \cite{MakinaCorpus}
	\item JSON Editor \cite{JeremyDorn}
	\item json-forms \cite{Brutusin.org}
	\item JSONForms \cite{EclipseSource}
	\item liform-react \cite{NachoMartin}
	\item Metawidget \cite{Metawidget}
	\item React JSON Schema Form \cite{MozillaServices}
\end{itemize}

En annan lösning till problemet är att använda ett separat JSON-objekt som fungerar som ett schema för formulärets utseende och beteende. Lösningarna som använder den sortens lösning är:
\begin{itemize}
	\item Angular Schema Form \cite{Textalk}
	\item Angular2 Schema Form \cite{MakinaCorpus}
	\item JSON Form \cite{Joshfire}
	\item json-forms \cite{Brutusin.org}
	\item JSONForms \cite{EclipseSource}
	\item liform-react \cite{NachoMartin}
	\item React JSON Schema Form \cite{MozillaServices}
	\item React Schema Form \cite{NetworkNewTechnologiesInc.}
\end{itemize}

Observera att vissa implementationer både utökar JSON Schemat, och lägger till extra scheman för att beskriva formulärets utseende och beteende. Vissa lösningar utökar schemat för att sedan lägga till ett extra mellansteg där vissa parametrar i modellen kopplas till extra funktionslogik. Inget verktyg ansågs vara enkelt att integrera med systemet då de skrevs för att fungera med hemsidor. Dessutom skrevs många verktyg för fel version av JSON Schema. Utöver det så använde nästan alla verktyg ett externt \textit{``ui-schema''} eller inkompatibla JSON Scheman. Därför ansågs inga verktyg vara tillräckliga och ett nytt skapades.

\section{Fördjupad utvärdering av grafiska användargränssnitt baserade på JSON Scheman}
\label{sec:forarbete:kategorisering}

De grafiska användargränssnitt som genererades av verktygen går att dela in i två kategorier vilket här kallas \textit{trädstruktur} samt \textit{platt formulär}. En \textit{trädstruktur} är ett grafiskt användargränssnitt som erbjuder en generell JSON-editor där datan visuellt representeras med en trädstruktur. Det finns många exempel på grafiska användargränssnitt som erbjuder ett grundläggande gränssnitt för att redigera generell JSON-data. De skiljer sig inte mycket mot att skriva JSON manuellt, men kan erbjuda enkel hjälp med nyckelord och formatering. Figur \ref{fig:json-editor} visar JSON Editor Online, vilket är ett exempel på ett sådant grafiskt användargränssnitt \cite{DeJong2018}. Det skulle kunna gå att begränsa vad användaren skulle kunna mata in med hjälp av JSON Schemat men ett sådant formulär har möjlighet att hantera JSON Data som följer all möjlig sorts struktur.

En stor fördel med \textit{trädstrukturer} är att då de speglar strukturen av ett JSON-dokument kan de enkelt användas för att visualisera nästan vilken sorts JSON-data som helst, oberoende datans komplexitet. En nackdel med att använda ett sådant grafiskt användargränssnitt är att det inte är enkla att anpassa för ändamålet, och det grafiska användargränssnittet blir inte särskilt frikopplat mot datan som ska manipuleras. Det blir dessutom svårt att presentera information till användaren som kan hjälpa användaren förstå hur datan kommer användas.

\begin{figure}
	\includegraphics[width=\textwidth]{./images/screenshot-json-editor.png}
	\vspace{-1.7em}
	\caption{JSON Editor Online \cite{DeJong2018}}
	\label{fig:json-editor}
\end{figure}

Ett annat alternativ är att göra som React JSON Schema Form (se figur \ref{fig:react-jsonschema-form}), vilket då hamnar i kategorin \textit{platt formulär}. React JSON Schema Form genererar ett webbformulär utifrån ett JSON Schema och ett extra JSON-dokument som kallas UISchema \cite{MozillaServices}. Varje datanod som kan manipuleras representeras med ett inmatningsfält och objekt och vektorer representeras med visuella grupperingar. Användargränssnittet blir mycket mer anpassat till syftet, och lättare att använda. Dessutom utnyttjas annoteringsnyckelorden \textit{title} och \textit{description} från JSON Schemat, för att förklara för användaren vad datan betyder.

\begin{figure}
	\includegraphics[width=\textwidth]{./images/screenshot-react-jsonschema-form.png}
	\vspace{-1.7em}
	\caption{React JSON Schema Form \cite{MozillaServices}}
	\label{fig:react-jsonschema-form}
\end{figure}

Det som är en nackdel hos \textit{platta formulär} är hur de presenterar djupt komplexa JSON-strukturer. Det är inte uppenbart hur ett grafiskt användargränssnitt skulle visualisera en vektor innehållande objekt med två \textit{properties}, varav ena \textit{propertyn} är ett objekt med två olika vektorer. I ett komplext formulär kan det också vara svårt att veta hur annoteringar som \textit{title} och \textit{description} borde presenteras. För att lösa problemet med representation av komplex data använder verktygen oftast ett externt \textit{``ui-schema''} eller tillägg till JSON Schemat för att tillföra extra annoteringsinformation om formulärets utseende. Det är en riktigt bra kompromiss om lösningen måste kunna klara av att presentera djupt komplexa datatyper, som inte följer en förutbestämd struktur, dock så uppfyller det inte kraven som ställdes på verktygen.

En \textit{trädstruktur} hanterar representation av djupt komplex data som \textit{platta formulär} inte klarar av lika bra. Det kan vara enklare att förstå den övergripande strukturen med en \textit{trädstruktur}, vilket speciellt gynnar datatyperna objekt och vektorer. Däremot bidrar \textit{platta formulär} mycket bättre förklaringar av enstaka datanoder som är någon av de andra datatyperna, vilket ger bättre förståelse för datanoderna, vilket \textit{trädstrukturer} inte lyckas med.
\input{./tex/5-resultat.tex}
\chapter{Diskussion, slutsats och fortsatt arbete}
\label{sec:slutsats}

%Arbetet har utförligt utvärderat JSON Schema, hur grafiska användargränssnitt kan genereras från scheman samt vad som saknas i JSON Scheman. Kapitel \ref{sec:slutsats:json-schema} diskuterar problemen som fortfarande finns med att använda JSON Scheman till att generera grafiska användargränssnitt. Kapitel \ref{sec:slutsats:fortsatt-arbete} föreslår framtida fokusområden att fortsätta arbetet med. Kapitel \ref{sec:slutsats:slutsats} diskuterar arbetet och resultatet av det, samt vilka slutsatser som kan dras efter arbetet.

Frågeställningarna som presenterades i kapitel \ref{sec:intro:problem} var:
\begin{itemize}
	\item Vilka svårigheter finns det med att använda JSON Schema för att automatisk generera ett grafiskt
	användargränssnitt?
	\item Är det möjligt?
	\item Hur generella JSON Scheman går det att använda sig av?
\end{itemize}
  
\section{Vilka svårigheter finns det med att använda JSON Schema för att automatisk generera ett grafiskt användargränssnitt?}
\label{sec:slutsats:json-schema}
JSON Schema är huvudsakligen ämnat för att validera data, trots att det erbjuder annoteringsfunktionalitet. Att generera ett grafiskt användargränssnitt från ett JSON Schema är därför inte trivialt. Att validera data kräver bara en uppsättning regler och en kontroll för att utvärdera om datan följer alla regler. Att generera ett interaktivt grafiskt användargränssnitt kräver att det finns ett tydligt sammanhang kring reglerna. Det grafiska användargränssnittet måste förstå hur data skapas för att passa in i reglerna, och ibland räcker inte regler för att förklara hur data ska skapas. Ett exempel på det är nyckelordet \textit{pattern} som används för att testa en textsträng mot ett ibland komplext mönster. Mönstret ger ingen enkel förklaring till hur en korrekt textsträng ser ut, utan ger bara en komplex uppsättning regler.

Nyckelorden \textit{allOf}, \textit{anyOf}, \textit{oneOf} och \textit{not} är ytterligare exempel på nyckelord som passar för att ställa upp en komplex uppsättning valideringsregler, men som försvårar för ett grafiskt användargränssnitt att försöka presentera sammanhanget av datan, och vilken data som är korrekt. Det är inte helt trivialt att presentera ett inmatningsfält för en datapunkt som får innehålla en \textit{(oneOf)} textsträng eller ett booleskt värde, men inte \textit{(not)} strängen \textit{"hello world"}.

Nyckelordet \textit{enum} saknar annoteringsmöjligheter för de förutbestämda alternativen i enumvektorn. Att introducera en till vektor som är frikopplad från \textit{enum} med annoteringar som ska tillhöra elementen i enumvektorn är ett alternativ som inte borde rekommenderas. Det går att skapa ett JSON Schema som är felaktigt på sättet att det felaktigt beskriver datan det ska beskriva, men att skriva ett schema som inte går att tolka, eller som kan tolkas på olika sätt ska inte vara möjligt. \citeauthor{Pezoa2016} utvärderar vissa JSON Scheman som kan tolkas olika av olika validerare. Ett av deras tester testade hur validerare tolkar cykliska referenser, vilket är tillåtet enligt JSON Schema men som är otolkbart av en validerare \cite{Pezoa2016}. Utöver det exemplet känner jag inte till några andra sätt att definiera ett korrekt JSON Schema som inte går att tolka. Om en frikopplad vektor lades till där varje element skulle tillhöra ett element i enumvektorn, skulle ytterligare ett sätt att skapa felaktiga JSON Scheman introduceras, om inte strikta specifikationer kring antalet element skulle specificeras.

Alternativet som jag rekommenderar och som projektet introducerar är att elementen i enumvektorn utökades med annoteringsfunktionalitet. Vilka annoteringar som hör till vilka enumvektorelemnt skulle inte kunna tolkas på mer än ett sätt, och det skulle inte begränsa nuvarande funktionalitet. Att kunna annotera enumvektorelementen är viktigt för att kunna generera bra grafiska användargränssnitt, där det grafiska användargränssnittet är frikopplat från datamodellen.

Ett annat område som borde arbetas på är att försöka erbjuda all funktionalitet som tidigare implementationer av grafiskt användargränssnittsgenerering erbjuder, utan att använda fler beskrivande dokument än ett JSON Schema eller utöka schemat med extra nyckelord. Nyckelordet \textit{format} skulle kunna användas mycket mer. Det skulle kanske kunna gå att lägga till ett generellt nyckelord som kan erbjuda valfri extra information som komplettering till \textit{format}. Det nyckelordet skulle inte behöva hantera validering, utan skulle bara användas för att lägga till information som \textit{format} inte täcker. Ett annat alternativ vore att tillåta \textit{format} att innehålla ett objekt med valfri struktur istället för bara en textsträng.

Vissa tidigare implementationer av genererade grafiska användargränssnitt, som exempelvis React JSON Schema Form \cite{MozillaServices}, erbjuder stöd för \textit{conditional schema dependencies} som implementerats på olika sätt men oftast inkompatibelt med JSON Schema specifikationerna. Det betyder att vissa delar av det grafiska användargränssnittet bara visas om någon eller några datapunkter uppfyller vissa krav. Kraven brukar vara strängare än valideringskraven för hela formuläret. Det kan exempelvis handla om att ett inmatningsfält representerar ett booleskt värde som ställer frågan \textit{``Har du någonsin köpt en telefon''}, och om användaren svarar ja så presenteras frågan \textit{``Hur många telefoner har du köpt?''} vilket är en fråga som bara är relevant om användaren svarade ja på första frågan. Nyckelorden \textit{if}, \textit{then} och \textit{else} lades nyss till JSON Schema specifikationerna på den senaste versionen, vilket kan förklara varför ingen av implementationerna har implementerat de än \cite{Andrews2018}. Det saknas arbete som utvärderar användbarheten hos de tre nyckelorden, vilket skulle kunna utvärdera om det finns brister med att använda dem till att generera ett grafiskt användargränssnitt eller om de är färdiga tillägg till JSON Schema.

\section{Är det möjligt att generera grafiska användargränssnitt utifrån generella JSON Scheman?}

Frågan är tvådelad och skulle kunna förtydligas. Ena sättet att tolka frågan är "Förutsatt att ett program får ett JSON Schema av okänd struktur, kan programmet alltid generera ett grafiskt användargränssnit som tillräckligt representerar informationen som JSON Scheman förmedlar?". Den andra tolkningen skulle kunna vara "Givet ett önskat grafiskt användargränssnitt, går det att skriva eller generera ett JSON Schema som sedan tolkas av ett program som korrekt genererar ett grafiskt användargränssnitt?". Ena frågan handlar om hur väl ett program hanterar olika JSON Scheman, medan den andra frågan handlar om hur bra JSON Scheman är för att beskriva grafiska användargränssnitt.

Arbetet föreslår en generell lösning som kan tillräckligt presentera ett grafiskt användargränssnitt utifrån alla sorts JSON Scheman. Arbetet handlade om att implementera en svagt specifik lösning men den visar samtidigt hur en generell lösning skulle se ut. Den tolkningen på frågan skulle enligt mig svaras med \textit{ja}. Det går att implementera program som kan hantera alla sorters JSON Scheman med en tidigare okänd struktur.

JSON Schema har däremot problem med att beskriva alla sorters önskade presentationsmöjligheter. Det är nästan perfekt men utan att nyckelorden \textit{enum} samt \textit{format} förbättras, är inte JSON Schema ett perfekt verktyg för att beskriva hur ett grafiskt användargränssnitt ska presentera manipulerbar data.


\section{Fortsatt arbete}
\label{sec:slutsats:fortsatt-arbete}
Ett intressant område för fortsatt arbete skulle vara att försöka generalisera lösningen som det här arbetet presenterade. Den vänstra trädstrukturen skulle kunna vara mer komplex med utökad funktionalitet, för att hantera objekt och vektorer, medan den högra sidan skulle kunna hantera textsträngar, siffror och booleska värden. Noderna i den vänstra trädstrukturen skulle kunna tydligare delas upp i objektnoder för att innehålla flera fält, och inställningsnoder eller lövnoder för att innehålla en enda enkel datastruktur. Dessutom skulle vissa noder kunna innehålla vektorer, med möjligheten att lägga till, ta bort och byta plats på element i vektorn. En svårighet skulle kunna vara hur ett element som får vara en av flera datatyper skulle hanteras.

Det här arbetet utvärderar inte utförligt användbarheten hos de olika typerna av grafiska användargränssnitt. Arbetet föreslår enbart en metod för att arbeta med generering av grafiska användargränssnitt som är beroende av ett JSON Schema. De olika typerna av grafiska användargränssnitt borde utvärderas i kvantitativa studier för att skapa en bättre förståelse för när de olika typerna av grafiska användargränssnitt ska borde användas.


\section{Slutsats}
\label{sec:slutsats:slutsats}
Arbetet lyckades med att generera ett grafiskt användargränssnitt på en skrivbordsapplikation, för att låta en användare manipulera data, som beskrevs med ett JSON Schema. Arbetet hanterade en förutbestämd struktur på JSON Scheman och JSON-dokument men har också föreslagit hur implementationen skulle kunna generaliseras. LSE kan enkelt utveckla Mimer SoftRadio till sina operatörsdatorer utan att behöva bry sig mycket om att synkronisera uppdateringar mellan administratörsprogrammen och operatörsdatorerna hos alla sina kunder. Administratörerna har erbjudits ett grafiskt användargränssnitt som är enkelt att använda, som tydligt förklarar alla möjliga inställningar och som förklarar hur inställningarna ska ställas in för att vara kompatibla med systemet.

Tyvärr bröt implementationen mot ett av de förutbestämda kraven, då nyckelordet \textit{enum} utökades, och att systemet därmed använde ett utökat JSON Schema som inte helt är kompatibelt med JSON Schema specifikationen. Kravbrottet erbjöd funktionalitet som inte skulle kunna varit möjlig annars, vilket är ett område för framtida utveckling av JSON Scheman.

JSON Schema fungerar utmärkt för validering men har brister när det gäller annotering. Det är förväntat i och med att JSON Schema inte är en färdig specifikation. De största bidragande faktorerna vore om \textit{format} utökades med att bidra mer generell annoteringsinformation än en textsträng, helst med \textit{format} som objekt, och utökade annoteringsmöjligheter av enumvektorelement

\printbibliography[heading=bibintoc] % Print the bibliography (and make it appear in the table of contents)

\appendix

\chapter{API-beskrivning}
\label{appendix:api-beskrivning}

\end{document}
