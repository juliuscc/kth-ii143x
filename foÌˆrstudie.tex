\documentclass[swedish]{kththesis}

\usepackage{blindtext} % This is just to get some nonsense text in this template, can be safely removed

\usepackage{csquotes} % Recommended by biblatex
\usepackage{biblatex}
\addbibresource{references.bib} % The file containing our references, in BibTeX format


\title{Detta är den svenska titeln}
\alttitle{This is the English translation of the title}
\author{Julius Recep Colliander Celik}
\email{jcelik@kth.se}
\supervisor{John Doe}
\examiner{Anders Västberg}
\principal{LS Elektronik AB}
\programme{Civilingenjör Informationsteknik}
\school{Skolan för elektroteknik och datavetenskap}
\date{\today}


\begin{document}

% Frontmatter includes the titlepage, abstracts and table-of-contents
\frontmatter

\titlepage

\begin{abstract}
  Svensk sammanfattning.

  \blindtext
\end{abstract}


\begin{otherlanguage}{english}
  \begin{abstract}
    English abstract.

    \blindtext
  \end{abstract}
\end{otherlanguage}


\tableofcontents


% Mainmatter is where the actual contents of the thesis goes
\mainmatter


\chapter{Introduktion}

% Vi använder paketet \emph{biblatex} för litteraturreferenser.  Därför anropar vi kommandot \texttt{parencite} för att få referenser inom parentes, så här \parencite{heisenberg2015}. Det är också möjligt att använda författarens namn som en del av en mening genom att använda \texttt{textcite}, om vi t.ex.\ talar om en studie av \textcite{einstein2016}.

\section{Problembeskrivning}

% Beskriv LS

% Beskriv Mimer

% Beskriv systemet

\section{Bakgrund}

\subsection{JSON och webkommunikation baserat på JSON objekt}
% Förklara JSON

\subsection{JSON Schema}
% Förklara JSON Schema

% Diskutera JSON Schema

% Förklara Delphi kanske??? Väldigt kanske

\section{Frågeställning}

Hur skulle man kunna använda en delmängd av JSON Schema för att dynamiskt anpassa ett användargränsnitt mot olika versioner av apier som erbjuder olika funkionalitet, på ett långsiktigt sätt?


\chapter{Metodval}

\begin{enumerate}
	\item Föreslå ett eget JSON Schema
	\item Skapa en JSON Schema genererare. ????
	\item Skapa en JSON Schema parser för Delphi. / Undersök befintliga JSON Schema parsers för Delphi.
	\item Skapa en direkt mappning mellan JSON Schema och en eller flera datatyp(er) i Delphi
	\item Create a dynamic interface based on parsed JSON Schemas.
\end{enumerate}

\section{Avgränsningar}

JSON Schema kan användas till:
\begin{itemize}
	\item validering av data
	\item anotering av data
	\item automatisk generering av kompatibel kod
	\item beskrivning av REST APIer
	\item automatisk generering av API-dokumentation
\end{itemize}

Dessutom finns det exempel på JSON Schema som automatiskt genereras utifrån kod XXXXX.

Detta projekt intresserar sig enbart för att försöka använda JSON Schema för att anotera data som kan redigeras. Det vill säga beskriva vilken data som kan redigeras, samt hur den kan redigeras. Därför kommer inte all funktionalitet av en JSON Parser implementeras, då det är utanför intresseområdet av rapporten.

Utöver funktionalitet kommer JSON parsern bara förstå en förenklad delmängd av JSON Schema, då vissa egenskaper av JSON Schema inte är eftertraktade. Ett exempel på ej eftertraktade egenskaper nyckelordet \texttt{multipleOf} är att kunna specificera att en \texttt{number} eller \texttt{integer} ska vara en multiple av en annan siffra.

Modellen som rapporten föreslår för att annotera JSON data kommer inte nödvändigtvis vara en strikt delmängd av JSON Schema. Om ej implementerade egenskaper behövs, kan modellen utökas för att inkludera egenskaper som saknas i JSON Schema.

\chapter{Resultat}

\section{Vad saknas i JSON Schema}
Hur hanterar man olika valideringsfel?

Föreslå kanske att JSON Schemat som föreslogs i rapport X ska användas.

\chapter{Diskussion och Slutsats}

\printbibliography[heading=bibintoc] % Print the bibliography (and make it appear in the table of contents)

\appendix

\chapter{Extra Material som Bilaga}

\end{document}
