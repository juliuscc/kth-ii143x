\chapter{Diskussion, slutsats och fortsatt arbete}
\label{sec:slutsats}
%Beskriv slutsatserna (reflektera över alla delar introduktionen i kapitel 1)
%
%Diskutera de positiva effekterna och nackdelarna. 
%
%Beskriv utvärderingen av resultaten av examensarbetet (om det inte beskrivits i resultatdelen)


\section{Att använda JSON Schema för att generera användargränssnitt}

JSON Schema är huvudsakligen ämnat för att validera data, trots att det erbjuder annoteringsfunktionalitet. Att generera ett användargränssnitt från ett JSON Schema är därför inte trivialt. Att validera data kräver bara en uppsättning regler och en kontroll för att utvärdera om datan följer alla regler. Att generera ett interaktivt användargränssnitt kräver att det finns ett tydligt samanhang kring reglerna. Användargränsnittet måste förstå hur data skapas för att passa in i reglerna, och ibland räcker inte regler för att förklara hur data ska skapas. Ett exempel på det är nyckelordet \textit{pattern} som används för att testa en textsträng mot ett ibland komplext mönster. Mönstret ger ingen enkel förklaring till hur en korrekt textsträng ser ut, utan ger bara en komplex upsättning regler.

Nyckelorden \textit{allOf}, \textit{anyOf}, \textit{oneOf} och \textit{not} är ytterligare exempel på nyckelord som passar för att ställa upp en komplex upsättning valideringsregler, men som försvårar för ett användargränssnitt att försöka presentera samanhanget av datan, och vilken data som är korrekt. Det är inte helt trivialt att presentera ett inmatningsfält för en datapunkt som får innehålla en \textit{(oneOf)} textsträng eller ett booleskt värde, men inte \textit{(not)} strängen \textit{"hello world"}.

Nyckelordet \textit{enum} saknar annoteringsmöjligheter för de förutbestämda alternativen i enumvektorn. Att introducera en till vektor som är frikopplad från \textit{enum} med annoteringar som ska tilhöra elementen i enumvektorn är ett alternativ som inte borde rekommenderas. Det går att skapa ett JSON Schema som är felaktigt på sättet att det felaktigt beskriver datan det ska beskriva, men att skriva ett schema som inte går att tolka, eller som kan tolkas på olika sätt ska inte vara möjligt. \citeauthor{Pezoa2016} utvärderar vissa JSON Scheman som kan tolkas olika av olika validerare. Ett av deras tester testade hur validerare tolkar cykliska referenser, vilket är tillåtet enligt JSON Schema men som är otolkbart av en validerare \cite{Pezoa2016}. Utöver det exemplet känner jag inte till några andra sätt att definera ett korrekt JSON Schema som inte går att tolka. Om en frikopplad vektor lades till där varje element skulle tillhöra ett element i enumvektorn, skulle ytterligare ett sätt att skapa felaktiga JSON Scheman introduceras, om inte strikta specifikationer kring antalet element skulle specificeras.

Alternativet som jag rekomenderar och som projektet introducerar är att elementen i enumvektorn utökades med annoteringsfunktionalitet. Vilka annoteringar som hör till vilka enumvektorelemnt skulle inte kunna tolkas på mer än ett sätt, och det skulle inte begränsa nuvarande funktionalitet. Att kunna annotera enumvektorelementen är viktigt för att kunna generera bra användargränssnitt, där användargränssnittet är frikopplat från datamodellen.

\section{Fortsatt arbete}
Ett intressant område för fortsatt arbete skulle vara att försöka generalisera lösningen som det här arbetet presenterade. Den vänstra trädstrukturen skulle kunna vara mer komplex med utökad funktionalitet, för att hantera objekt och vektorer, medan den högra sidan skulle kunna hantera textsträngar, siffror och booleska värden. Noderna i den vänstra trädstrukturen skulle kunna tydligare delas upp i objektnoder för att innehålla flera fält, och inställningsnoder eller lövnoder för att innehålla en enda enkel datastruktur. Dessutom skulel vissa noder kunna inehålla vektorer, med möjligheten att lägga till, ta bort och byta plats på element i vektorn. En svårighet skulle kunna vara hur ett element som får vara en av flera datatyper skulle hanteras.

Det här arbetet utvärderar inte utförligt användbarheten hos de olika typerna av grafiska användargränssnitt. Arbetet föreslår enbart en metod för att arbeta med generering av grafiska användargränssnitt som är beroende av ett JSON Schema. De olika typerna av grafiska användargränssnitt borde utvärderas i kvantitativa studier för att skapa en bättre förståelse för när de olika typerna av grafiska användargänssnitt ska borde användas.

Ett annat område som borde arbetas på är att försöka erbjuda all funktionalitet som tidigare implementationer av användargränssnittgenerering erbjuder, utan att använda fler beskrivande dokument än ett JSON Schema eller utöka schemat med extra nyckelord. Nyckelordet \textit{format} skulle kunna användas mycket mer. Det skulle kanske kunna gå att lägga till ett generellt nyckelord som kan erbjuda valfri extra information som kompletering till \textit{format}. Det nyckelordet skulle inte behöva hantera validering, utan skulle bara användas för att lägga till information som \textit{format} inte täcker. Ett annat alternativ vore att tillåta \textit{format} att innehålla ett objekt istället för bara en textsträng.

Vissa tidigare implementationer av genererade användargränssnitt, som exempelvis React JSON Schema Form \cite{MozillaServices}, erbjuder stöd för \textit{conditional schema dependencies} som implementerats på olika sätt men oftast inkompatibelt med JSON Schema specifikationerna. Det betyder att vissa delar av användargränssnittet bara visas om någon eller några datapunkter upfyller vissa krav. Kraven brukar vara strängare än valideringskraven för hela formuläret. Det kan exempelvis handla om att ett inmatningsfält representerar ett booleskt värde som ställer frågan \textit{``Har du någonsin köpt en telefon''}, och om användaren svarar ja så presenteras frågan \textit{``Hur många telefoner har du köpt?''} vilket är en fråga som bara är relevant om användaren svarade ja på första frågan. Nyckelorden \textit{if}, \textit{then} och \textit{else} lades nyss till JSON Schema specifikationerna på den senaste verisonen, vilket kan förklara att ingen av implementatoinerna har implementerat de än \cite{Andrews2018}. Det saknas arbete som utvärderar användbarheten hos de tre nyckelorden, vilket skulle kunna utvärdera om det finns brister med att använda dem till att generera användargränssnitt eller om de är färdiga tillägg till JSON Schema.

\section{Slutsats}

Arbetet lyckades med att generera användargränssnitt på en skrivbordsapplikation, för att låta en användare manipulera data, som beskrevs med ett JSON Schema. Arbetet hanterade en förutbestämd struktur på JSON Scheman och JSON-dokument men har också föreslagit hur implementationen skulle kunna generaliseras. LSE kan enkelt utveckla Mimer SoftRadio till sina operatörsdatorer utan att behöva bry sig mycket om att synkronisera uppdateringar mellan administratörsprogrammen och operatörsdatorerna hos alla sina kunder. Administratörerna har erbjudits ett användargränssnitt som är enkelt att använda, som tydligt förklarar alla möjliga inställningar och som förklarar hur inställningarna ska ställas in för att vara kompatibla med systemet.

JSON Schema fungerar utmärkt för validering men har brister när det gäller annotering. Det är förväntat i och med att JSON Schema inte är en färdig specifikation. De största bidragande faktorerna vore om \textit{format} utökades med att bidra mer generell annoteringsinformation än en sträng, helst med \textit{format} som objekt, och utökade annoteringsmöjligheter av enumvektorelement