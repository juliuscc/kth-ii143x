\chapter{Introduktion}
Examensarbetet handlade om att undersöka möjligheten att skapa en modell för att beskriva och annotera redigerbar data, och sedan automatiskt generera ett användargränssnitt. Det här kapitlet introducerar hela arbetet. Kapitel \ref{sec:intro:bakgrund} ger en bakgrund till arbetet, vilket innefattar både en enkel teoretisk bakgrund, samt en bakgrund till företaget arbetet utfördes hos, samt systemet som arbetet utvecklades mot. Kapitel \ref{sec:intro:problemområde} diskuterar systemet arbetet utvecklades mot i större detalj. Kapitel \ref{sec:intro:problem} presenterar problemet med en frågeställning. Kapitel \ref{sec:intro:hypotes} föreslår en hypotes som arbetet grundar sig på. Kapitel \ref{sec:intro:syfte} och \ref{sec:intro:mål} diskuterar syftet och målet med arbetet. Riskerna diskuteras i Kapitel \ref{sec:intro:risker}. Kapitel \ref{sec:intro:metodval} presenterar metodvalet. Kapitel \ref{sec:intro:avgränsningar} beskriver arbetets omfattning. Resten av rapportens disposition presenteras i kapitel \ref{sec:intro:disposition}.

\section{Bakgrund}
\label{sec:intro:bakgrund}
Det här kapitlet beskriver bakgrunden till varför och i vilka ämnesområden arbetet utfördes, samt för att ge en förståelse för resten av Introduktionen. Kapitel \ref{sec:intro:json} beskriver JSON och JSON Schema vilket är en stor del av teknikerna som arbetet undersökte. Kapitel \ref{sec:intro:mimer} beskriver företaget arbetet utfördes hos, samt systemet som arbetet utvecklades mot.

\subsection{JSON och JSON Schema}
\label{sec:intro:json}
JavaScript Object Notation \textit{(JSON)} är ett textbaserat dataformat för att utbyta data mellan webbtjänster. Till skillnad mot andra alternativ, som exempelvis XML, är det både läsbart för människor och datorer, samtidigt som det är väldigt kompakt, vilket är en anledning till att det är ett av de mest populäraste dataformaten för datautbyte mellan webbtjänster. \cite{Pezoa2016} JSON utvecklades med inspiration till språket ECMAScript \textit{(JavaScript)} men samtidigt programmeringsspråksoberoende, vilket lett till att implementationer för att generera och parsa JSON finns tillgängliga i många olika programmeringsspråk \cite{ECMA2013}. ECMAScript är ett språk som stöds av alla moderna webbläsare, och har därför blivit en kärnteknik för webben.

Trots att JSON är det populäraste dataformatet för datautbyte mellan webbtjänster saknas det ett väletablerat standardiserat ramverk för metadata-definition \cite{Pezoa2016}. En väldigt lovande formell standard är JSON Schema, vilket är ett ramverk som fortfarande utvecklas av Internet Engineering Task Force \textit{(IETF)}. JSON Schema är ett ramverk för att beskriva och annotera JSON-data \cite{A.Wright}. Kapitel \ref{sec:teori} beskriver JSON och JSON Schema i mer detalj.

\subsection{Mimer SoftRadio}
\label{sec:intro:mimer}
Arbetet utfördes hos LS Elektronik AB \textit{(LSE)}, som är ett tekniskt företag, som utvecklar och tillverkar elektroniska produkter \cite{Ehne}. LSE erbjuder bland annat ett radiosystem som heter Mimer SoftRadio vilket kan användas för att ansluta ett flertal annars inkompatibla radioenheter i ett och samma system, samt fjärrstyra radioenheterna från en persondator med ett klientprogram. I resten av rapporten kan datorn med klientprogrammet kallas operatörsdator, där användaren kan kallas operatör.

Mimer SoftRadio är ett program med väldigt många möjliga inställningar. I många fall är dessa inställningar för komplexa för de vanliga operatörerna att redigera själva, så därför brukar vissa kunder låsa redigeringsmöjligheterna, och bara tillåta vissa administratörer att ställa in alla inställningar på rätt sätt. Det finns också kundfall där flera operatörer använder samma dator, vid olika tidpunkter. Ett förekommande kundfall är att en operatör jobbade dagtid med att leda och organisera dagsarbete, medan en annan operatör tar över nattskiftet för att övervaka många fler radioenheter.

För att förenkla dessa två kundfall påbörjade LSE utvecklingen av funktionalitet som skulle erbjuda användare att spara uppsättningar av inställningar i olika \textit{profiler}. Det skulle gå att enkelt byta mellan flera förinställda konfigurationer av Mimer SoftRadio. För att konfigurera dessa profiler skapades ett administratörsprogram, som skulle kunna fjärrkonfigurera profilerna hos operatörsdatorerna. Fjärrstyrningen skulle underlätta administratörer att ställa in profiler på flera datorer samtidigt, som sannolikt skulle innehålla liknande inställningar.

\section{Problemområde}
\label{sec:intro:problemområde}
Systemet för att konfigurera profiler, som beskrevs i kapitel \ref{sec:intro:mimer}, bygger på att alla operatörsdatorer exponerar ett API mot en server, över en TCP-port. Administratörsprogrammet skulle erbjuda ett användargränssnitt för att konfigurera profilinställningar, för att sedan kommunicera ändringarna till operatörsdatorerna. I resten av rapporten kan operatörsdatorerna och administratörsprogrammet kallas server respektive klient.

Kommunikationsprotokollet var ett eget skapat protokoll som byggde på att skicka JSON-Objekt via TCP. För en beskrivning av JSON, se kapitel \ref{sec:teori:json}. För en mer utförlig beskrivning av kommunikationsprotokollet se Appendix \ref{appendix:api-beskrivning}. Problemet som LSE hade inför utvecklandet av användarprofilerna var skapandet av ett användargränssnitt på administratörsprogrammet. Olika operatörsdatorer, hos olika kunder, kunde ha olika versioner av Mimer, med olika funktionalitet tillgänglig, och därmed olika uppsättningar konfigurerbara inställningar. 

Det vore orimligt kostsamt för LSE att skapa ett administratörsprogram för varje version av Mimer, då både Mimer ändrades med tiden, samt att olika kunder köpte till extra funktionalitet. Samtidigt behövde användargränssnittet på administratörsprogrammet anpassas så att det skulle vara tydligt vad en administratör kunde konfigurera. Helst skulle ett administratörsprogram fungera bra med framtida versioner av Mimer, utan några eller utan stora justeringar av programmet. LSE ville helt enkelt att servern kommunicerade tillgängliga inställningar, till klienten så att klienten sen skulle kunna anpassa sitt användargränssnitt, och det skulle ske på ett tillräckligt generellt sätt att administratörsprogrammet var framtidssäkert för framtida version av Mimer.

\section{Problem}
\label{sec:intro:problem}
Vilka svårigheter finns det med att använda JSON Schema för att automatisk generera ett användargränssnitt, är det möjligt, samt hur generella JSON Scheman går det att använda sig av?

\section{Hypotes}
\label{sec:intro:hypotes}
En förutsatt hypotes sattes innan arbetet påbörjades. Hypotesen var att det borde vara möjligt att automatiskt skapa ett användargränssnitt från ett JSON Schema, men för att skapa ett användarvänligt användargränssnitt var det nödvändigt att klienten utvecklades för ett JSON Schema som delvis följde en förutbestämd struktur.

\section{Syfte}
\label{sec:intro:syfte}
Syftet med tesen är att systematiskt analysera problemen med att försöka skapa automatiskt genererade användargränssnitt utifrån olika JSON Scheman. Målet är att föreslå både en strukturell modell samt en metod för att lösa detta problem.

Syftet med arbetet är att med hjälp av JSON Scheman skapa ett dynamiskt användargränssnitt som anpassade sig efter syfte. Utan att uppdatera administratörsklienten ska samma administratörsklient kunna konfigurera inställningar hos olika datorer med olika versioner av Mimer SoftRadio, och därmed olika uppsättningar konfigurerbara inställningar. Det skulle inte bara innebära stark kompatibilitet utan också framtidssäkerhet hos LSEs produkter.

\section{Mål}
\label{sec:intro:mål}
Målet med arbetet är att kunna skapa en grund för användare av Mimer SoftRadio att enkelt kunna konfigurera inställningar, oavsett version eller uppsättning extra funktionalitet. Det skulle kunna innebära att Mimer SoftRadio blir ett bättre verktyg för många potentiella kunder. Samtidigt finns det ett mål med att utforska samt att metodiskt utvärdera och beskriva hur användargränssnitt för att redigera data, automatiskt kan genereras.

\subsection{Samhällsnytta, Etik och Hållbarhet}

Ur ett samhällsnyttigt och etiskt perspektiv kan en implementation av fjärrstyrda användarprofiler i Mimer SoftRadio innebära effektivisering av samhällsnyttiga funktioner. Mimer SoftRadio används bland annat av polis, ambulans, brandkår, kollektivtrafik och internationella flygplatser. Fjärrstyrda användarprofiler skulle hos befintliga kunder i många fall innebära effektivare arbete. Som följd av detta går det att argumentera för att det leder till effektivare kommunikation för organisationer som använder Mimer SoftRadio. Då dessa organisationer arbetar med säkerhet i samhället, räddandet av liv, och upprätthållandet av ett effektivt samhälle, lider hela samhället när dessa organisationer inte kan kommunicera ordentligt. Det är därför väldigt etiskt försvarbart att arbeta med att effektivisera arbetet hos dessa organisationer.

\section{Risker}
\label{sec:intro:risker}
En ekonomisk risk är risken som alltid finns vid all hantering av data. Om någon datahantering skulle bli fel, och data skulle försvinna, skrivas över eller bli korrupt, så måste kunder tillägna tid åt att återskapa datan. Därför är det viktigt att implementera ett robust system som är delvis feltolerant. Ingen data som kommer hanteras kommer vara kritisk, och kommer vara relativt enkel att återskapa. Problemet blir att det skulle innebära en kostnad att behöva skapa profiler igen och ställa in inställningar igen, och utöver det så skulle korrupta filer till och med kunna innebära att Mimer SoftRadio inte går att använda alls.

\section{Metodval}
\label{sec:intro:metodval}
Arbetet som ska utföras är till viss del en fallstudie, men samtidigt ska den utforska något nytt och med det föreslå en ny modell. Designorienterad forskning \textit{(Design science research)} är den metod som passar bäst för den här sortens arbete och därför har den metoden valts.
\noindent
Arbetet följde de följande stegen:

\begin{enumerate}
	\item \textbf{Medvetenhet} En beskrivning av problemet som ska lösas med modellen.
	\item \textbf{Förslag} Förslag på lösning presenteras.
	\item \textbf{Utveckling} Modellen utvecklas.
	\item \textbf{Utvärdering} Modellen utvärderas. Lyckades modellen lösa problemet beskrivet i \textit{Medvetenhet}?
	\item \textbf{Sammanfattning} Dra slutsatser
\end{enumerate}

\section{Avgränsningar}
\label{sec:intro:avgränsningar}
En viktig avgränsning är att rapporten endast väldigt ytligt kommer undersöka olika användargränssnitt, och användbarheten hos dem. Arbetet handlar inte primärt om användbarhet, utan arbetet handlar i större grad om hur JSON Schema kan automatisera skapandet av användbara gränssnitt. Med hjälp av kunskapen som arbetet presenterar kan användbara användargränssnitt enklare skapas.

Ett annat ämne som också är viktigt är säkerhet av systemet som skapas. Säkerheten hos applikationen omfattas inte av arbetet, men det ignoreras samtidigt inte. Systemet som utvecklas för att utbyta JSON Scheman och JSON-data sker över en ssl-krypterad säker uppkoppling. Det här arbetet utvärderar inte säkerheten hos den uppkopplingen.

Att skapa ett användarvänligt användargränssnitt utifrån alla möjliga sorters JSON Scheman med samma verktyg omfattas inte av arbetet. Arbetet kommer utforska olika strategier och metoder för att arbeta med förutbestämda JSON Scheman.

Validering av data är något som webbtjänster ofta måste ta hänsyn till. En klient kan annars skicka otillåten data till en webbserver och därför måste webbservern alltid validera data när den tar emot data, innan data används eller lagras. JSON Scheman fungerar utmärkt för validering av data, men då datan valideras hos klienten, både klienten och servern omfattas av arbetet, samt att användarna inte anses ha uppsåt att förstöra eller falsifiera data, kommer JSON Scheman inte användas för att validera data hos servern.

\section{Disposition}
\label{sec:intro:disposition}
Kapitel \ref{sec:teori} presenterar den teoretiska bakgrunden.
