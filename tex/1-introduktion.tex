\chapter{Introduktion}

Examensarbetet handlar om att annotera manipulerbar data, och sedan automatiskt generera ett grafiskt användargränssnitt för att interagera med datan. Vad jag känner till är detta det första publikt publicerade försöket av att generera ett grafiskt användargränssnitt från JSON Schema, som var implementerat i något annat programmeringsspråk än HTML, CSS eller JavaScript. Detta är relevant då JSON Scheman utvecklats med hänsyn till JSON och JavaScript, samtidigt som många andra språk och miljöer skulle kunna dra nytta av funktionaliteten som JSON Schema kan erbjuda. Det är viktigt att JSON Schema fungerar bra med andra språk än JavaScript för att det ska vara ett bra och användbart verktyg.

\section{Bakgrund}
\label{sec:intro:bakgrund}
Det här kapitlet beskriver bakgrunden till varför och i vilka ämnesområden arbetet utfördes, samt för att ge en förståelse för resten av introduktionen. JSON och JSON Schema utgör en stor del av teknikerna som arbetet undersökte. LS Elektronik AB \textit{(LSE)} är företaget arbetet utfördes hos, och Mimer SoftRadio är systemet som arbetet utvecklades mot.

\subsection{JSON och JSON Schema}
\label{sec:intro:json}
JavaScript Object Notation \textit{(JSON)} är ett textbaserat dataformat för att utbyta data mellan webbtjänster. Till skillnad mot andra alternativ, som exempelvis XML, är det både läsbart för människor och datorer, samtidigt som det är väldigt kompakt, vilket är en anledning till att det är ett av de mest populäraste dataformaten för datautbyte mellan webbtjänster \cite{Pezoa2016}. JSON utvecklades med inspiration till språket ECMAScript \textit{(JavaScript)} men är samtidigt oberoende av programmeringsspråk, vilket lett till att implementationer för att generera och parsa JSON finns tillgängliga i många olika programmeringsspråk \cite{ECMA2013}. ECMAScript är ett språk som stöds av alla moderna webbläsare, och har därför blivit en kärnteknik för webben.

Trots att JSON är det populäraste dataformatet för datautbyte mellan webbtjänster saknas det ett väletablerat standardiserat ramverk för metadata-definition \cite{Pezoa2016}. En väldigt lovande formell standard är JSON Schema, vilket är ett ramverk som fortfarande utvecklas av Internet Engineering Task Force \textit{(IETF)}. JSON Schema är ett ramverk för att beskriva och annotera JSON-data \cite{A.Wright}. Kapitel \ref{sec:teori} beskriver JSON och JSON Schema i mer detalj.

\subsection{Mimer SoftRadio}
\label{sec:intro:mimer}
Arbetet utfördes hos LSE, som är ett teknikföretag, som utvecklar och tillverkar elektroniska produkter \cite{Ehne}. LSE erbjuder bland annat ett radiosystem som heter Mimer SoftRadio vilket kan användas för att ansluta ett flertal annars inkompatibla radioenheter i ett och samma system, samt fjärrstyra radioenheterna från en persondator med ett klientprogram. I resten av rapporten kan datorn med klientprogrammet kallas operatörsdator, där användaren kan kallas operatör.

Mimer SoftRadio \textit{Mimer} är ett program med väldigt många möjliga inställningar. I många fall är dessa inställningar för komplexa för de vanliga operatörerna att redigera själva, så därför brukar vissa kunder låsa redigeringsmöjligheterna, och bara tillåta vissa administratörer att ställa in alla inställningar på rätt sätt. Det finns också kundfall där flera operatörer använder samma dator, vid olika tidpunkter. Ett förekommande kundfall är att en operatör jobbar dagtid med att leda och organisera dagsarbete, medan en annan operatör tar över nattskiftet för att övervaka många fler radioenheter.

För att förenkla dessa två kundfall påbörjade LSE utvecklingen av funktionalitet som skulle erbjuda användare att spara uppsättningar av inställningar i olika \textit{profiler}. Det skulle gå att enkelt byta mellan flera förinställda konfigurationer av Mimer. För att konfigurera dessa profiler skapades ett administratörsprogram, som skulle kunna fjärrkonfigurera profilerna hos operatörsdatorerna. Fjärrstyrningen skulle underlätta administratörer att ställa in profiler på flera datorer samtidigt, som sannolikt skulle innehålla liknande inställningar.

\section{Problemområde}
\label{sec:intro:problemområde}
Systemet för att konfigurera profiler (se kapitel \ref{sec:intro:mimer}) bygger på att alla operatörsdatorer exponerar ett API mot en server, över en TCP-port. Administratörsprogrammet ska erbjuda ett grafiskt användargränssnitt för att konfigurera profilinställningar, för att sedan kommunicera ändringarna till operatörsdatorerna. I resten av rapporten kan operatörsdatorerna och administratörsprogrammet kallas server respektive klient. Administrationsprogrammet skrivs som en skrivbordsapplikation till Windows, med språket Delphi.

Kommunikationsprotokollet är ett eget skapat protokoll som bygger på att skicka JSON-objekt via TCP. För en beskrivning av JSON, se kapitel \ref{sec:teori:json}. För en mer utförlig beskrivning av kommunikationsprotokollet se bilaga \ref{appendix:api-beskrivning}. Problemet som LSE har inför utvecklandet av användarprofilerna är skapandet av ett grafiskt användargränssnitt på administratörsprogrammet. Olika operatörsdatorer, hos olika kunder, kan ha olika versioner av Mimer, med olika funktionalitet tillgänglig, och därmed olika uppsättningar konfigurerbara inställningar.

Det vore orimligt kostsamt för LSE att skapa olika administratörsprogram för varje version av Mimer, då både Mimer ändras med tiden, samt att olika kunder köper till extra funktionalitet. Samtidigt måste det grafiska användargränssnittet på administratörsprogrammet anpassas så att det ska vara tydligt vad en administratör kan konfigurera. Helst ska ett administratörsprogram fungera bra med framtida versioner av Mimer, utan några eller utan stora justeringar av programmet. LSE vill helt enkelt att servern kommunicerar tillgängliga inställningar till klienten, så att klienten sen ska kunna anpassa sitt grafiska användargränssnitt, och det ska ske på ett så tillräckligt generellt sätt att administratörsprogrammet är framtidssäkert för framtida versioner av Mimer.

\section{Problem}
\label{sec:intro:problem}
Vilka svårigheter finns det med att använda JSON Schema för att automatisk generera ett grafiskt användargränssnitt? Är det möjligt? Hur generella JSON Scheman går det att använda sig av?

Det finns även mer praktiska problem som uppstår vid konstruktionen av arkitekturen för systemet som ska utvecklas. De är:

\begin{itemize}
	\item Hur ska användarinteraktion integreras i detta system, så att interaktion faktiskt manipulerar data?
	\item Hur ska schemat som formuläret baseras på, genereras?
	\item Hur ska schemat parsas, tolkas och sedan representeras i språket Delphi?
	\item Hur ska schemats representation tillsammans med data presenteras i ett grafiskt användargränssnitt?
\end{itemize}

\section{Syfte}
\label{sec:intro:syfte}
Syftet med rapporten är att systematiskt analysera problemen med att försöka skapa automatiskt genererade grafiska användargränssnitt utifrån olika JSON Scheman. Syftet med arbetet är att med hjälp av JSON Scheman skapa ett dynamiskt grafiskt användargränssnitt som anpassade sig efter syfte. Utan att uppdatera administratörsklienten ska samma administratörsklient kunna konfigurera inställningar hos olika datorer med olika versioner av Mimer, och därmed olika uppsättningar konfigurerbara inställningar. Det skulle inte bara innebära stark kompatibilitet utan också framtidssäkerhet hos LSEs produkter.

\section{Mål}
\label{sec:intro:mål}
Målet med arbetet är att kunna skapa en grund för användare av Mimer att enkelt kunna konfigurera inställningar, oavsett version eller uppsättning extra funktionalitet. Det skulle kunna innebära att Mimer blir ett bättre verktyg för många potentiella kunder. Samtidigt finns det ett mål med att utforska samt att metodiskt utvärdera och beskriva hur grafiska användargränssnitt för att redigera data, automatiskt kan genereras.

\subsection{Samhällsnytta, Etik och Hållbarhet}

Ur ett samhällsnyttigt och etiskt perspektiv kan en implementation av fjärrstyrda användarprofiler i Mimer SoftRadio innebära effektivisering av samhällsnyttiga funktioner. Mimer användas bland annat av polis, ambulans, brandkår, kollektivtrafik och internationella flygplatser. Fjärrstyrda användarprofiler skulle hos befintliga kunder i många fall innebära effektivare arbete. Som följd av detta går det att argumentera för att det leder till effektivare kommunikation för organisationer som använder Mimer. Då dessa organisationer arbetar med säkerhet i samhället, räddandet av liv, och upprätthållandet av ett effektivt samhälle, lider hela samhället när dessa organisationer inte kan kommunicera ordentligt. Det är därför väldigt etiskt försvarbart att arbeta med att effektivisera arbetet hos dessa organisationer.

\section{Risker}
\label{sec:intro:risker}
En ekonomisk risk är risken som alltid finns vid all hantering av data. Om någon datahantering skulle bli fel, och data skulle försvinna, skrivas över eller bli korrupt, så måste kunder tillägna tid åt att återskapa datan. Därför är det viktigt att implementera ett robust system som är delvis feltolerant. Ingen data som kommer hanteras kommer vara kritisk, och kommer vara relativt enkel att återskapa. Problemet blir att det skulle innebära en kostnad att behöva skapa profiler igen och ställa in inställningar igen, och utöver det så skulle korrupta filer till och med kunna innebära att Mimer SoftRadio inte går att använda alls.

\section{Metodval}
\label{sec:intro:metodval}
Arbetet var av kvalitativ karaktär. Arbetet handlade om att utforska JSON Scheman som är en ny utvecklande teknik. Specifikt utforskades förmågan att använda JSON Scheman för annoteringar som sedan används för att generera grafiska användarformulär för att låta användare manipulera den data som annoterades. 

Det saknas stora datamängder som krävs för att genomföra statistisk analys, som är kärnan av kvantitativa studier. Dessutom är JSON Schema som ämne fortfarande outforskat, och speciellt att använda JSON Scheman för att generera grafiska användargränssnitt för att manipulera datan som schemat beskriver.

Problemet som arbetet grundade sig på var en fallstudie. Ett specifikt problem skulle lösas utifrån vissa krav och en ny, oprövad och ofärdig teknik skulle användas för att lösa problemet. Det fanns vissa befintliga lösningar som utvärderades utifrån perspektivet av fallstudien. Både de befintliga lösningarna utvärderades, samt hur användbart JSON Schema som verktyg är för att lösa problemet som presenterades i fallstudien.

För att utvärdera om någon av lösningarna löste fallstudien, utvärderades alla befintliga lösningar utifrån förbestämda krav. Samtidigt implementerades ett system för att testa tekniken JSON Scheman. Beslutet togs att befintliga verktyg skulle delvis eller helt användas om det var möjligt utifrån de förbestämda kraven.

\section{Avgränsningar}
\label{sec:intro:avgränsningar}
En viktig avgränsning är att rapporten endast väldigt ytligt undersöker olika grafiska användargränssnitt, och användbarheten hos dem. Arbetet handlar inte primärt om användbarhet, utan arbetet handlar i större grad om hur JSON Schema kan automatisera skapandet av användbara gränssnitt. Med hjälp av kunskapen som arbetet presenterar kan användbara grafiska användargränssnitt enklare skapas.

Ett annat ämne som också är viktigt är säkerhet av systemet som skapas. Säkerheten hos applikationen omfattas inte av arbetet, men det ignoreras samtidigt inte. Systemet som utvecklas för att utbyta JSON Scheman och JSON-data sker över en SSL-krypterad säker uppkoppling. Det här arbetet utvärderar inte säkerheten hos den uppkopplingen.

Att skapa ett användbart grafiskt användargränssnitt utifrån alla möjliga sorters JSON Scheman med samma verktyg omfattas inte av arbetet. Arbetet kommer utforska olika strategier och metoder för att arbeta med förutbestämda JSON Scheman. Dessutom kommer bara en delmängd av JSON Scheman att hanteras, och en färdig JSON Schema parser skapas och testas ej.

Validering av data är något som webbtjänster ofta måste ta hänsyn till. En klient kan annars skicka otillåten data till en webbserver och därför måste webbservern alltid validera data när den tar emot data, innan data används eller lagras. Ett stort användningsområde av JSON Schema är validering men det omfattas inte av arbetet. Valideringsdata i JSON Schema kommer endast användas för att extrapolera annoteringsinformation. Anledningen till att validering inte omfattas är att inmatningsdata valideras hos klienten, samt att både klienten och servern omfattas av arbetet. En till faktor är att användarna inte anses ha uppsåt att förstöra eller falsifiera data.

\section{Disposition}
\label{sec:intro:disposition}
Kapitel \ref{sec:teori} presenterar den teoretiska bakgrunden. Kapitel \ref{sec:metod} beskriver metodologin som valdes. Kapitel \ref{sec:forarbete} presenterar och utvärderade kända implementationer som berörde liknande ämnesområden som det här arbetet. Kapitel \ref{sec:arbetet} redovisar hur systemet utvecklades, vilka val som togs, varför de valen togs samt diskuterar hur arbetet förhåller sig till tidigare implementationer av liknande projekt och verktyg. Kapitel \ref{sec:slutsats} diskuterar hur bra JSON Schema är som verktyg för att annotera manipulerbar data, vilka outforskade intresseområden som skulle kunna behöva fortsatt arbete, samt utvärderar arbetets resultat.