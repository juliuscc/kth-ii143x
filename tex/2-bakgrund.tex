\chapter{Teoretisk Bakgrund OBS BYT UT!}
\label{sec:teori}

\section{JSON och webkommunikation baserat på JSON objekt}
\label{sec:teori:json}
JSON erbjuder stöd för några enkla datastrukturer: textsträngar \textit{(string)} (\mintinline{json}{"hej värld"}), tal \textit{(number)} (\mintinline{json}{4}), ett tomt värde \textit{(null)} (\mintinline{json}{null}), samt booleska värden \textit{(booleans)} (\mintinline{json}{false}). JSON erbjuder dessutom stöd för två komplexa datatyper vilket är vektorer \textit{(array)}, en ordnad lista av JSON-värden vilket visas i Figur \ref{fig:json-array-example}, samt objekt \textit{(object)}, vilket är en oordnad mängd av namn-värde-par, samt visas i Figur \ref{fig:json-object-example}.

\begin{figure}
	\begin{minted}[tabsize=2, frame=single, fontsize=\small, framesep=2mm]{json}
["hej värld", 4, null, false]
	\end{minted}
	\vspace{-1.7em}
	\caption{Exempel på JSON-array}
	\label{fig:json-array-example}
\end{figure}

\begin{figure}
	\begin{minted}[tabsize=2, frame=single, fontsize=\small, framesep=2mm]{json}
{
	"firstName": "Erik",
	"lastName": "Andersson",
	"age": 30
}
	\end{minted}
	\vspace{-1.7em}
	\caption{Exempel på JSON-object}
	\label{fig:json-object-example}
\end{figure}

\noindent
Med hjälp av att rekursivt använda \textit{array} eller \textit{object} går det att representera komplexa datastrukturer med hjälp av JSON. Det finns inga begränsningar i hur komplext datastrukturer kan representeras.

\section{JSON i webbkomunikation}
\label{sec:teori:json-web}
På grund av att JSON är kompakt, enkelt läsbart och har brett stöd hos många språk och implementationer, har JSON blivit väldigt utbrett bland webbtjänster. En hypotetisk förfrågan till en webbtjänst skulle kunna se ut som i Figur \ref{fig:json-request-example}, där en klient förfrågar om de nuvarande väderförhållandena i Stockholm i Sverige. Svaret från webbservern skulle kunna se ut som i Figur \ref{fig:json-response-example} där webbservern svarar att temperaturen är minus tre grader celsius och att det snöar. Exemplet visar hur simpelt JSON som dataformat är att förstå, vilket skulle kunna vara en delvis förklaring för populariteten.

\begin{figure}
	\begin{minted}[tabsize=2, frame=single, fontsize=\small, framesep=2mm]{json}
{
	"country": "Sweden",
	"city": "Stockholm"
}
\end{minted}
	\vspace{-1.7em}
	\caption{Exempel på förfrågan till webbserver}
	\label{fig:json-request-example}
\end{figure}

\begin{figure}[h]
	\begin{minted}[tabsize=2, frame=single, fontsize=\small, framesep=2mm]{json}
{
	"timestamp": "06/01/2018 10:45:08",
	"country": "Sweden",
	"city": "Stockholm",
	"weather": "Snowing",
	"temperature": -3
}
	\end{minted}
	\vspace{-1.7em}
	\caption{Exempel på svar på förfrågan från webbserver}
	\label{fig:json-response-example}
\end{figure}

\section{JSON Schema}


JSON Schema är ett ramverk för att förklara hur JSON värden kan se ut. JSON Schema specifiserar regler som kan användas för att antingen bestämma om befintliga JSON värden är giltiga, eller för att i förväg beskriva hur gilltiga värden får se ut. 

Objektet i Figur \ref{fig:json-object-example} skulle kunna valideras enligt följande JSON Schema som visas i Figur \ref{fig:json-schema-example}. Användningsområden för detta är bland annat:

\begin{figure}
	\begin{minted}[tabsize=2, frame=single, fontsize=\small, framesep=2mm]{json}
		{
			"type": "object",
			"required": ["firstName", "age"],
			"properties": {
				"firstName": { "type": "string" },
				"lastName": { "type": "string" },
				"age": { "type": "integer" }
			}
		}
	\end{minted}
	\vspace{-1.7em}
	\caption{Exempel på simpelt JSON Schema}
	\label{fig:json-schema-example}
\end{figure}

% HYPER SCHEMA??

\begin{enumerate}
	\item Validering av data.
	\item Annotering av data.
	\item Beskrivning av REST APIer.
	\item Automatisk generering av kompatibel kod, för att hantera JSON värden beskrivna med JSON Schema.
	\item Automatisk generering av API-dokumentation.
	\item Automatisk generering av användargränsnitt.
\end{enumerate}

\noindent
Att använda JSON Schema för användningsområdena 1-3 är trivialt. Det går att utveckla program som kan hantera alla oändligt möjliga permutationer av JSON Schema. Det som däremot inte är trivialt är hur användningsområdena 4-6 skulle kunna generaliseras så pass mycket att ett program eller algoritm skulle kunna hantera vilket giltigt JSON Schema som helst. Användningsområde fyra och fem omfattas inte av den här rapporten, och varför användningsområde sex inte är trivialt diskuteras mer i kapitel XXX.

% Diskutera JSON Schema mer? Förklara reglerna?

